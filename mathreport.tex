% ╔═══════════════════════════════════════════════════════════════════════════╗
% ║                        MindFlow 功能演示文档                               ║
% ║                       Report 模式 + 审稿 + 水印                            ║
% ╠═══════════════════════════════════════════════════════════════════════════╣
% ║  文件: mathreport.tex                                                      ║
% ║  说明: 演示 review 审稿模式、水印功能、批注系统                            ║
% ║  编译: xelatex mathreport.tex                                              ║
% ╚═══════════════════════════════════════════════════════════════════════════╝
\documentclass[linux, report, review]{mindflow}
%
% ┏━━━━━━━━━━━━━━━━━━━━━━━━━━━━━━━━━━━━━━━━━━━━━━━━━━━━━━━━━━━━━━━━━━━━━━━━━━━┓
% ┃                           📋 类选项完整参考                               ┃
% ┗━━━━━━━━━━━━━━━━━━━━━━━━━━━━━━━━━━━━━━━━━━━━━━━━━━━━━━━━━━━━━━━━━━━━━━━━━━━┛
%
% ┌─────────────────────────────────────────────────────────────────────────┐
% │ 【文档模式】三选一                                                       │
% ├─────────────────────────────────────────────────────────────────────────┤
% │  note   │ 基于 ctexart  │ 笔记模式,\section 为顶层                     │
% │  book   │ 基于 ctexbook │ 书籍模式,支持 \chapter                       │
% │  report │ 基于 ctexrep  │ 报告模式 ✓,\section 呈现大标题样式           │
% └─────────────────────────────────────────────────────────────────────────┘
%
% ┌─────────────────────────────────────────────────────────────────────────┐
% │ 【平台适配】linux ✓ / mac / win                                          │
% │ 【审稿模式】review ✓ → 启用行号显示,方便批注定位                        │
% └─────────────────────────────────────────────────────────────────────────┘
%
% ┌─────────────────────────────────────────────────────────────────────────┐
% │ 【Section 样式】十选一                                                   │
% ├───────────────────────┬─────────────────────────────────────────────────┤
% │  secstyle-classic     │  蓝色编号块 + 浅蓝背景 (默认)                   │
% │  secstyle-modern      │  大号编号 + 渐变底边线                          │
% │  secstyle-minimal     │  极简:左侧竖线 + 纯文本                        │
% │  secstyle-boxed       │  边框卡片 + 阴影                                │
% │  secstyle-neon        │  🟢 霓虹发光 (赛博朋克)                         │
% │  secstyle-terminal    │  🖥️ 终端风格 (Hacker)                           │
% │  secstyle-gradient    │  🌈 渐变背景 (科技感)                           │
% │  secstyle-elegant     │  ✨ 金色装饰 (学术典雅)                         │
% │  secstyle-blueprint   │  📐 蓝图网格 (工程师)                           │
% │  secstyle-ribbon      │  🎗️ 折叠丝带 (精致卡片)                         │
% └───────────────────────┴─────────────────────────────────────────────────┘
%
% ┏━━━━━━━━━━━━━━━━━━━━━━━━━━━━━━━━━━━━━━━━━━━━━━━━━━━━━━━━━━━━━━━━━━━━━━━━━━━┓
% ┃                            💧 水印功能详解                                ┃
% ┗━━━━━━━━━━━━━━━━━━━━━━━━━━━━━━━━━━━━━━━━━━━━━━━━━━━━━━━━━━━━━━━━━━━━━━━━━━━┛
%
% 【文本水印】 \mfWatermarkText[颜色][缩放]{文字}
%   参数: 颜色(可选,默认gray!25) | 缩放(可选,默认4) | 文字(必需)
%
%   示例:
%     \mfWatermarkText{DRAFT}              % 默认灰色草稿
%     \mfWatermarkText[red!15]{绝密}       % 红色绝密
%     \mfWatermarkText[blue!10][6]{内部}   % 蓝色大号
%
% 【图片水印】 \mfWatermarkImage{图片路径}
%   示例: \mfWatermarkImage{figure/logo.png}
%
% 【默认】水印默认不显示,调用上述命令后才启用
%
% ┏━━━━━━━━━━━━━━━━━━━━━━━━━━━━━━━━━━━━━━━━━━━━━━━━━━━━━━━━━━━━━━━━━━━━━━━━━━━┓
% ┃                            📝 批注系统说明                                ┃
% ┗━━━━━━━━━━━━━━━━━━━━━━━━━━━━━━━━━━━━━━━━━━━━━━━━━━━━━━━━━━━━━━━━━━━━━━━━━━━┛
%
%   \todo{内容}     → 🟠 待办:橙色边框,标记未完成任务
%   \fixme{内容}    → 🔴 修正:红色边框,标记错误或问题
%   \notebox{内容}  → 🔵 备注:蓝色边框,补充说明
%
%   💡 配合 review 模式的行号,可精确定位批注位置
%
% ═══════════════════════════════════════════════════════════════════════════

\title{实验报告:一维热方程的数值解法}
\author{MindFlow 实验员}
\date{\today}

% 启用文本水印演示
\mfWatermarkText[gray!15][2.5]{CJX}

\begin{document}

\maketitle
\mftableofcontents

% =========================================================
\section{实验目的}
% =========================================================

\todo{待补充:增加实验的工程意义和应用场景描述}

本实验的目标:
\begin{enumerate}
    \item 理解抛物型 PDE 的差分格式原理
    \item 实现显式欧拉格式 (FTCS) 与隐式格式 (Backward Euler)
    \item 验证数值解的稳定性条件与收敛阶
    \item 比较不同格式的计算效率
\end{enumerate}

% =========================================================
\section{问题描述}
% =========================================================

求解如下一维热传导方程初边值问题:
\begin{align}
    \pd{u}{t} &= \alpha \pdd{u}{x}, \quad 0 < x < L, \quad t > 0 \\
    u(x, 0) &= \sin(\pi x / L) \quad \text{(初始条件)} \\
    u(0, t) &= u(L, t) = 0 \quad \text{(Dirichlet 边界条件)}
\end{align}

\begin{defnnew}{精确解}
    利用分离变量法可得精确解:
    \[
        u(x, t) = e^{-\alpha (\pi/L)^2 t} \sin(\pi x / L)
    \]
    这是一个指数衰减的正弦波,衰减率由热扩散系数 $\alpha$ 决定。
\end{defnnew}

\fixme{第 2.1 节的推导过程需要补充完整的分离变量法步骤}

\newpage
% =========================================================
\section{数值方法}
% =========================================================

\subsection{显式格式 (FTCS)}

\begin{center}
	\papyrus{%
		\begin{minipage}{0.7\textwidth}
			\textbf{麦克斯韦方程组}\\[0.3em]
			$\nabla \cdot \vec{E} = \frac{\rho}{\varepsilon_0}$, \quad
			$\nabla \cdot \vec{B} = 0$\\[0.2em]
			$\nabla \times \vec{E} = -\frac{\partial \vec{B}}{\partial t}$, \quad
			$\nabla \times \vec{B} = \mu_0 \vec{J} + \mu_0 \varepsilon_0 \frac{\partial \vec{E}}{\partial t}$
		\end{minipage}
	}
\end{center}

对时间采用向前差分,空间采用中心差分:
\[
    \frac{u_i^{n+1} - u_i^n}{\Delta t} = \alpha \frac{u_{i+1}^n - 2u_i^n + u_{i-1}^n}{\Delta x^2}
\]

令网格比 $r = \frac{\alpha \Delta t}{\Delta x^2}$,迭代公式为:
\[
    u_i^{n+1} = r u_{i+1}^n + (1 - 2r) u_i^n + r u_{i-1}^n
\]

\begin{warning}{稳定性条件}
    FTCS 格式是条件稳定的,Von Neumann 稳定性分析表明必须满足:
    \[
        r = \frac{\alpha \Delta t}{\Delta x^2} \le \frac{1}{2}
    \]
    否则数值解将产生震荡并发散。这严格限制了时间步长的选取。
\end{warning}

\subsection{隐式格式 (Backward Euler)}

\notebox{隐式格式无条件稳定,但需要求解线性方程组,计算量更大}

对时间采用向后差分:
\[
    \frac{u_i^{n+1} - u_i^n}{\Delta t} = \alpha \frac{u_{i+1}^{n+1} - 2u_i^{n+1} + u_{i-1}^{n+1}}{\Delta x^2}
\]

整理得三对角方程组 $A \vect{u}^{n+1} = \vect{u}^n$,可用 Thomas 算法高效求解。

% =========================================================
\section{实验结果}
% =========================================================

\subsection{稳定性验证}

选取参数 $L=1$, $\alpha=0.01$, $\Delta x = 0.1$,测试不同网格比的效果。

\begin{figurerow}{不同网格比下的数值解}[2]
    \figitem{figure/py2.png}{$r=0.4$ (稳定)}
    \figitem{figure/py3.png}{$r=0.6$ (发散)}
\end{figurerow}

\subsection{误差分析}

\begin{theoremnew}{截断误差}
    FTCS 格式的局部截断误差为 $O(\Delta t) + O(\Delta x^2)$,即时间一阶、空间二阶精度。
\end{theoremnew}

下表记录了不同时空步长下的最大误差 $E_\infty = \max_{i} |u_i^N - u_{exact}(x_i)|$:

\begin{table}[H]
    \centering
    \caption{误差收敛表}
    \begin{tabular}{cccc}
        \toprule
        $\Delta x$ & $\Delta t$ & $E_\infty$ & 收敛阶 \\
        \midrule
        0.100 & 0.0010 & 1.23e-3 & - \\
        0.050 & 0.00025 & 3.05e-4 & 2.01 \\
        0.025 & 0.00006 & 7.61e-5 & 2.00 \\
        0.0125 & 0.000015 & 1.90e-5 & 2.00 \\
        \bottomrule
    \end{tabular}
\end{table}

\begin{tip}{误差分析技巧}
    收敛阶可用公式 $p = \log_2(E_h / E_{h/2})$ 计算。若结果接近 2,说明空间二阶精度得到验证。
\end{tip}

% =========================================================
\section{结论与讨论}
% =========================================================

\begin{conclusion}{实验结论}
    \begin{enumerate}
        \item \textbf{FTCS 格式}:实现简单、计算高效,但受 $r \le 0.5$ 稳定性条件限制
        \item \textbf{误差验证}:实验结果验证了 $O(\Delta x^2 + \Delta t)$ 的理论截断误差阶
        \item \textbf{发散现象}:在网格比 $r > 0.5$ 时确实观察到数值发散,与理论分析一致
        \item \textbf{隐式格式}:无条件稳定,适用于刚性问题,但计算成本更高
    \end{enumerate}
\end{conclusion}

\todo{后续工作:实现 Crank-Nicolson 格式 (时间二阶精度) 并进行对比实验}

% =========================================================
\section{附录:核心代码}
% =========================================================

\begin{codeblock}[python]{FTCS 格式实现}
import numpy as np

def heat_ftcs(L, T, nx, nt, alpha):
    """
    使用 FTCS 格式求解一维热方程
    
    Parameters:
        L: 空间域长度
        T: 时间域长度  
        nx: 空间网格点数
        nt: 时间步数
        alpha: 热扩散系数
    
    Returns:
        u: 最终时刻的数值解
    """
    dx = L / (nx - 1)
    dt = T / (nt - 1)
    r = alpha * dt / dx**2
    
    # 稳定性检查
    if r > 0.5:
        print(f"Warning: r={r:.3f} > 0.5, 格式不稳定!")
    
    # 初始条件
    x = np.linspace(0, L, nx)
    u = np.sin(np.pi * x / L)
    
    # 时间推进
    for n in range(1, nt):
        u_new = u.copy()
        u_new[1:-1] = r*u[2:] + (1-2*r)*u[1:-1] + r*u[:-2]
        u = u_new
    
    return u
\end{codeblock}

\begin{codeblock}[python]{误差计算与可视化}
import matplotlib.pyplot as plt

def compute_error(u_num, u_exact):
    """计算最大范数误差"""
    return np.max(np.abs(u_num - u_exact))

# 参数设置
L, T, alpha = 1.0, 0.1, 0.01
nx, nt = 101, 1001

# 数值解
u_num = heat_ftcs(L, T, nx, nt, alpha)

# 精确解
x = np.linspace(0, L, nx)
u_exact = np.exp(-alpha * (np.pi/L)**2 * T) * np.sin(np.pi * x / L)

# 误差
error = compute_error(u_num, u_exact)
print(f"Max error: {error:.6e}")
\end{codeblock}

% =========================================================
% 演示:取消水印
% =========================================================
% 如果要取消水印,可在文档末尾添加以下命令:
% \DraftwatermarkOptions{stamp=false}

\end{document}
