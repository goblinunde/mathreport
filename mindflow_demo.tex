% ╔═══════════════════════════════════════════════════════════════════════════╗
% ║                    MindFlow 核心功能演示文档                               ║
% ║                         Complete Features Demo                             ║
% ╠═══════════════════════════════════════════════════════════════════════════╣
% ║  文件: mindflow_demo.tex                                                   ║
% ║  说明: 系统性展示 mindflow.cls 的所有功能,作为使用参考手册                ║
% ║  编译: xelatex mindflow_demo.tex                                           ║
% ╚═══════════════════════════════════════════════════════════════════════════╝
\documentclass[note, linux, secstyle-classic]{mindflow}
% 💡 尝试切换 secstyle-[classic|modern|minimal|boxed|neon|terminal|gradient|elegant|blueprint|ribbon]

\title{MindFlow 核心功能演示}
\author{MindFlow 用户}
\date{\today}

% 水印演示 (可取消注释启用)
% \mfWatermarkText[gray!10][4]{DEMO}

\begin{document}

\maketitle
\mftableofcontents  % 目录页罗马数字,正文从1开始

% ═══════════════════════════════════════════════════════════════════════════
\section{类选项概览}
% ═══════════════════════════════════════════════════════════════════════════

\begin{notice}{文档模式}
    MindFlow 提供三种文档模式:
    \begin{itemize}
        \item \texttt{note} (默认) - 基于 ctexart,适合日常笔记
        \item \texttt{book} - 基于 ctexbook,适合系统教材
        \item \texttt{report} - 基于 ctexrep,适合实验报告
    \end{itemize}
\end{notice}

\subsection{Section 样式一览}

MindFlow 提供 \textbf{10 种 Section 样式},可通过类选项切换:

\begin{table}[H]
    \centering
    \caption{Section 样式对照表}
    \begin{tabular}{lll}
        \toprule
        \textbf{选项} & \textbf{类别} & \textbf{视觉效果} \\
        \midrule
        \texttt{secstyle-classic} & 基础 & 蓝色编号块 + 浅蓝背景 (默认) \\
        \texttt{secstyle-modern} & 基础 & 大号编号 + 渐变底边线 \\
        \texttt{secstyle-minimal} & 基础 & 左侧竖线 + 纯文本 \\
        \texttt{secstyle-boxed} & 基础 & 边框卡片 + 阴影 \\
        \midrule
        \texttt{secstyle-neon} & 极客 & 🟢 霓虹发光 (赛博朋克) \\
        \texttt{secstyle-terminal} & 极客 & 🖥️ 终端命令行风格 \\
        \texttt{secstyle-gradient} & 极客 & 🌈 蓝紫渐变背景 \\
        \texttt{secstyle-elegant} & 极客 & ✨ 金色装饰线 (学术典雅) \\
        \texttt{secstyle-blueprint} & 极客 & 📐 蓝图网格背景 \\
        \texttt{secstyle-ribbon} & 极客 & 🎗️ 折叠丝带效果 \\
        \bottomrule
    \end{tabular}
\end{table}

\subsection{其他选项}

\begin{itemize}
    \item \texttt{linux} / \texttt{mac} / \texttt{win} - 平台字体配置
    \item \texttt{review} - 启用行号显示 (审稿模式)
    \item \texttt{chapnum} / \texttt{nochapnum} - 章节编号 / 全局连续编号
\end{itemize}

% ═══════════════════════════════════════════════════════════════════════════
\section{数学定理环境}
% ═══════════════════════════════════════════════════════════════════════════

\subsection{原版 amsthm 环境}

\begin{theorem}[Lax-Milgram 定理]
    设 $V$ 是 Hilbert 空间,$a(\cdot,\cdot)$ 是 $V$ 上的有界强制双线性形式,$F \in V'$。
    则存在唯一的 $u \in V$ 使得 $a(u, v) = F(v), \; \forall v \in V$。
\end{theorem}

\begin{proof}
    利用 Riesz 表示定理构造算子 $A: V \to V$,证明其为双射。
\end{proof}

\subsection{美化版定理环境 (tcolorbox)}

\begin{theoremnew}{梯度下降收敛定理}
    设 $f: \R^n \to \R$ 是 $L$-光滑凸函数,若使用步长 $\eta = \frac{1}{L}$ 的梯度下降:
    \[
        x_{k+1} = x_k - \eta \nabla f(x_k)
    \]
    则有 $f(x_k) - f(x^*) \leq \frac{L \norm{x_0 - x^*}^2}{2k}$。
\end{theoremnew}

\begin{proofnew}
    由 $L$-光滑性,对任意 $x, y$:
    \[
        f(y) \leq f(x) + \nabla f(x)^\top (y - x) + \frac{L}{2} \norm{y - x}^2
    \]
    令 $y = x_{k+1}$,代入整理即得。
\end{proofnew}

\begin{defnnew}{Sobolev 空间 $W^{k,p}$}
    设 $\Omega \subset \R^n$ 是开集,$k \in \N$,$1 \le p \le \infty$。定义
    \[
        W^{k,p}(\Omega) := \Set{ u \in L^p(\Omega) \st D^\alpha u \in L^p(\Omega), \; |\alpha| \le k }
    \]
    其中 $D^\alpha u$ 是弱导数。
\end{defnnew}

\begin{lemmanew}{Grönwall 引理}
    若 $y(t) \le a(t) + \int_0^t b(s) y(s) \ds$,则
    $y(t) \le a(t) + \int_0^t a(s) b(s) e^{\int_s^t b(r) \dd r} \ds$。
\end{lemmanew}

\begin{corollarynew}{能量估计}
    对于抛物方程,解的 $L^2$ 范数随时间指数衰减。
\end{corollarynew}

\begin{examplenew}{热方程初边值问题}
    考虑域 $\Omega$ 上的热方程:
    \[
        \pd{u}{t} - \lap u = 0, \quad u|_{\partial\Omega} = 0, \quad u(x,0) = u_0(x)
    \]
\end{examplenew}

\begin{remarknew}
    美化版环境与原版环境共享计数器,编号格式为 section-编号。
\end{remarknew}

% ═══════════════════════════════════════════════════════════════════════════
\section{提示框环境}
% ═══════════════════════════════════════════════════════════════════════════

\begin{notice}{关于 physics 包}
    MindFlow 加载了 physics 包,提供 \verb|\pdv|, \verb|\grad|, \verb|\norm| 等便捷命令。
\end{notice}

\begin{tip}{性能优化建议}
    \begin{enumerate}
        \item 使用 mini-batch 梯度下降平衡效率与稳定性
        \item 添加学习率预热 (warmup) 避免初期震荡
        \item 使用梯度裁剪防止梯度爆炸
    \end{enumerate}
\end{tip}

\begin{warning}{数值稳定性}
    计算 $\softmax$ 时应先减去最大值防止溢出:
    \[
        \softmax(z_i) = \frac{e^{z_i - \max(z)}}{\sum_j e^{z_j - \max(z)}}
    \]
\end{warning}

\begin{conclusion}{本节要点}
    \begin{itemize}
        \item notice - 信息提示 (蓝色)
        \item tip - 技巧建议 (绿色)
        \item warning - 警告注意 (橙色)
        \item conclusion - 总结归纳 (紫色)
    \end{itemize}
\end{conclusion}

% ═══════════════════════════════════════════════════════════════════════════
\section{导读与总结框}
% ═══════════════════════════════════════════════════════════════════════════

\begin{introbox}[本节导读]
    导读框适合放在章节开头,概述本节内容和学习目标。支持自定义标题。
\end{introbox}

\begin{summarybox}[本节小结]
    \begin{enumerate}
        \item 总结框适合放在章节结尾
        \item 用于归纳核心知识点
        \item 帮助读者复习回顾
    \end{enumerate}
\end{summarybox}

% ═══════════════════════════════════════════════════════════════════════════
\section{数学宏库}
% ═══════════════════════════════════════════════════════════════════════════

\subsection{微分与积分算子}

\begin{defbox}{微分与积分宏}
    \begin{itemize}
        \item 偏导数: $\pd{u}{t}$, $\pdd{u}{x}$ (二阶)
        \item 微分元: $\dx$, $\dy$, $\dt$, $\dmu$ (测度)
        \item 区域积分: $\intO f \dx$, $\intRn f \dx$
        \item 边界积分: $\intpO g \ds$
    \end{itemize}
\end{defbox}

热方程的经典形式:
\[
    \pd{u}{t} = \alpha \pdd{u}{x}, \quad x \in \Omega, \quad t > 0
\]

\subsection{函数空间与范数}

\begin{itemize}
    \item Sobolev 空间: $\Wkp{1}{p}(\Omega)$, $\Hk[1](\Omega)$
    \item $L^p$ 空间: $\Lp(\Omega)$, $\Lp[2](\Omega)$
    \item 范数: $\normL{f}$, $\normL[2]{f}$
\end{itemize}

\subsection{常用集合与收敛}

\begin{itemize}
    \item 数集: $\R$ (实数), $\N$ (自然数), $\Z$ (整数), $\Q$ (有理数), $\C$ (复数)
    \item 收敛: $u_n \weakto u$ (弱收敛), $X \embeds Y$ (嵌入)
    \item 极限: $\limn a_n$, $\limx[0^+] f(x)$
\end{itemize}

\subsection{深度学习符号}

\begin{defbox}{深度学习符号表}
    \begin{itemize}
        \item 基础: 损失 $\loss$, 数据集 $\data$, 模型 $\model$, 参数 $\param$
        \item 概率: 期望 $\expect[\cdot]$, 概率 $\prob(\cdot)$, KL散度 $\KLdiv(P\|Q)$
        \item 激活: $\relu(x)$, $\sigmoid(x)$, $\tanhfunc(x)$, $\softmax(z)$
        \item 网络: $\Conv(x)$, $\LSTM(x)$, $\BatchNorm(x)$, $\Dropout(x)$
        \item 向量: $\vect{x}$, $\weight$, $\bias$
    \end{itemize}
\end{defbox}

% ═══════════════════════════════════════════════════════════════════════════
\section{代码环境}
% ═══════════════════════════════════════════════════════════════════════════

\begin{codeblock}[python]{PyTorch 神经网络示例}
import torch
import torch.nn as nn

class SimpleNet(nn.Module):
    def __init__(self, input_dim, hidden_dim, output_dim):
        super().__init__()
        self.fc1 = nn.Linear(input_dim, hidden_dim)
        self.relu = nn.ReLU()
        self.fc2 = nn.Linear(hidden_dim, output_dim)
    
    def forward(self, x):
        x = self.relu(self.fc1(x))
        return self.fc2(x)

# 训练循环
model = SimpleNet(784, 256, 10)
optimizer = torch.optim.Adam(model.parameters(), lr=1e-3)
\end{codeblock}

\begin{codeblock}[tex]{LaTeX 定理环境使用}
\begin{theoremnew}{定理名称}
    定理内容...
\end{theoremnew}

\begin{proofnew}
    证明过程...
\end{proofnew}
\end{codeblock}

% ═══════════════════════════════════════════════════════════════════════════
\section{图文混排环境}
% ═══════════════════════════════════════════════════════════════════════════

\subsection{textfigure 环境}

图文并排环境,支持左右布局:

\begin{textfigure}[right]{figure/py2.png}{训练曲线示例}
    神经网络的训练过程可以通过损失函数的下降曲线来监控。理想情况下,训练损失和验证损失应该同时下降并趋于稳定。
    
    如果验证损失在某个点开始上升而训练损失继续下降,则说明模型出现了过拟合。此时可以考虑:增加训练数据、使用正则化 (L2, Dropout)、提前停止训练。
\end{textfigure}

\subsection{parallelfigures 环境}

并排双图环境:

\begin{parallelfigures}{训练过程对比}
    \addfig[0.45]{figure/py2.png}{训练损失}
    \addfig[0.45]{figure/py3.png}{验证损失}
\end{parallelfigures}

\subsection{figurerow 环境}

多图行环境,支持自定义列数:

\begin{figurerow}{不同学习率的训练效果}[3]
    \figitem{figure/py2.png}{$\eta=0.001$}
    \figitem{figure/py3.png}{$\eta=0.01$}
    \figitem{figure/py2.png}{$\eta=0.1$}
\end{figurerow}

\subsection{figuregrid 环境}

网格布局环境:

\begin{figuregrid}{模型架构对比}
    \gridrow{%
        \gridfig[0.45]{figure/py2.png}{ResNet}%
        \gridfig[0.45]{figure/py3.png}{VGG}%
    }
\end{figuregrid}

% ═══════════════════════════════════════════════════════════════════════════
\section{科学绘图 (PGFPlots)}
% ═══════════════════════════════════════════════════════════════════════════

MindFlow 提供 \textbf{6 种预设绘图样式},满足不同场景需求:

\begin{table}[H]
    \centering
    \caption{PGFPlots 样式一览}
    \begin{tabular}{lll}
        \toprule
        \textbf{样式} & \textbf{用途} & \textbf{特点} \\
        \midrule
        \texttt{mfplot} & 通用折线图 & 网格 + 彩色标记 (默认) \\
        \texttt{mfplot-dark} & 暗色主题 & 黑底 + 霓虹配色 \\
        \texttt{mfplot-minimal} & 极简风格 & 无网格 + 箭头坐标轴 \\
        \texttt{mfplot-bar} & 柱状图 & 分类数据可视化 \\
        \texttt{mfplot-scatter} & 散点图 & 数据分布展示 \\
        \texttt{mfplot-area} & 面积图 & 堆叠区域展示 \\
        \bottomrule
    \end{tabular}
\end{table}

\subsection{默认样式 (mfplot)}

\begin{figure}[H]
    \centering
    \begin{tikzpicture}
        \begin{axis}[
            mfplot,
            xlabel={Epoch},
            ylabel={Loss},
            title={训练损失曲线},
            legend entries={Train, Validation}
        ]
            \addplot coordinates {
                (1, 2.5) (2, 1.8) (3, 1.4) (4, 1.1) (5, 0.9)
                (6, 0.8) (7, 0.75) (8, 0.72) (9, 0.70) (10, 0.69)
            };
            \addplot coordinates {
                (1, 2.8) (2, 2.1) (3, 1.7) (4, 1.5) (5, 1.4)
                (6, 1.35) (7, 1.32) (8, 1.30) (9, 1.29) (10, 1.28)
            };
        \end{axis}
    \end{tikzpicture}
    \caption{mfplot: 标准折线图,适合训练曲线展示}
\end{figure}

\subsection{暗色主题 (mfplot-dark)}

\begin{figure}[H]
    \centering
    \begin{tikzpicture}
        \begin{axis}[
            mfplot-dark,
            xlabel={Time (s)},
            ylabel={Amplitude},
            title={Signal Analysis}
        ]
            \addplot[domain=0:10, samples=100] {sin(deg(x))};
            \addplot[domain=0:10, samples=100] {cos(deg(x))};
            \legend{$\sin(x)$, $\cos(x)$}
        \end{axis}
    \end{tikzpicture}
    \caption{mfplot-dark: 暗色主题,适合演示文稿}
\end{figure}

\subsection{极简样式 (mfplot-minimal)}

\begin{figure}[H]
    \centering
    \begin{tikzpicture}
        \begin{axis}[
            mfplot-minimal,
            xlabel={$x$},
            ylabel={$f(x)$},
            domain=-2:2,
            samples=50
        ]
            \addplot {x^2};
            \addplot {x^3};
            \addplot {exp(x)};
            \legend{$x^2$, $x^3$, $e^x$}
        \end{axis}
    \end{tikzpicture}
    \caption{mfplot-minimal: 极简风格,适合教材插图}
\end{figure}

\subsection{柱状图样式 (mfplot-bar)}

\begin{figure}[H]
    \centering
    \begin{tikzpicture}
        \begin{axis}[
            mfplot-bar,
            xlabel={Model},
            ylabel={Accuracy (\%)},
            ymin=0, ymax=100
        ]
            \addplot coordinates {(A, 85) (B, 78) (C, 92) (D, 88) (E, 95)};
            \addplot coordinates {(A, 82) (B, 75) (C, 89) (D, 85) (E, 91)};
            \legend{Train, Test}
        \end{axis}
    \end{tikzpicture}
    \caption{mfplot-bar: 柱状图,适合模型对比}
\end{figure}

\subsection{散点图样式 (mfplot-scatter)}

\begin{figure}[H]
    \centering
    \begin{tikzpicture}
        \begin{axis}[
            mfplot-scatter,
            xlabel={Feature 1},
            ylabel={Feature 2},
            title={数据分布可视化}
        ]
            \addplot+[mark=*, blue!70] coordinates {
                (1,2) (1.5,2.5) (2,3) (2.2,2.8) (1.8,2.2) (2.5,3.5) (3,4)
            };
            \addplot+[mark=triangle*, red!70] coordinates {
                (4,1) (4.5,1.5) (5,2) (4.8,1.8) (5.2,2.2) (5.5,2.5)
            };
            \legend{Class A, Class B}
        \end{axis}
    \end{tikzpicture}
    \caption{mfplot-scatter: 散点图,适合分类数据展示}
\end{figure}

\begin{tip}{PGFPlots 使用建议}
    \begin{itemize}
        \item 所有样式可通过 \verb|[style, key=value]| 进一步自定义
        \item 使用 \verb|\addplot+[...]| 可覆盖默认样式
        \item 复杂图形建议使用 \verb|\pgfplotstableread| 读取外部数据
    \end{itemize}
\end{tip}

% ═══════════════════════════════════════════════════════════════════════════
\section{TikZ 绘图库}
% ═══════════════════════════════════════════════════════════════════════════

MindFlow 提供了丰富的 TikZ 预设样式,用于绘制神经网络、图论、流程图、拓扑空间等。

\begin{table}[H]
    \centering
    \caption{TikZ 样式分类}
    \begin{tabular}{lll}
        \toprule
        \textbf{类别} & \textbf{样式} & \textbf{用途} \\
        \midrule
        通用节点 & \texttt{mfnode}, \texttt{mfbox}, \texttt{mfdiamond} & 基础图形节点 \\
        神经网络 & \texttt{neuron}, \texttt{input neuron}, \texttt{nnedge} & 网络结构图 \\
        图论 & \texttt{graphedge}, \texttt{dirgraphedge}, \texttt{edgelabel} & 图与路径 \\
        流程图 & \texttt{process}, \texttt{decision}, \texttt{terminal} & 算法流程 \\
        状态机 & \texttt{state}, \texttt{transition}, \texttt{loop edge} & 有限自动机 \\
        拓扑 & \texttt{open set}, \texttt{closed set}, \texttt{point} & 拓扑空间 \\
        \bottomrule
    \end{tabular}
\end{table}

\subsection{神经网络结构图}

\begin{figure}[H]
    \centering
    \begin{tikzpicture}[x=1.8cm, y=1.2cm]
        % 输入层
        \foreach \i in {1,2,3} {
            \node[input neuron] (I\i) at (0, -\i) {$x_\i$};
        }
        % 隐藏层
        \foreach \j in {1,2,3,4} {
            \node[hidden neuron] (H\j) at (2, -\j+0.5) {};
        }
        % 输出层
        \foreach \k in {1,2} {
            \node[output neuron] (O\k) at (4, -\k-0.5) {$y_\k$};
        }
        % 连接
        \foreach \i in {1,2,3} {
            \foreach \j in {1,2,3,4} {
                \draw[nnedge] (I\i) -- (H\j);
            }
        }
        \foreach \j in {1,2,3,4} {
            \foreach \k in {1,2} {
                \draw[nnedge] (H\j) -- (O\k);
            }
        }
        % 标签
        \node[above=5pt] at (0, -0.5) {\small 输入层};
        \node[above=5pt] at (2, 0) {\small 隐藏层};
        \node[above=5pt] at (4, -1) {\small 输出层};
    \end{tikzpicture}
    \caption{全连接神经网络结构 (使用 neuron 样式)}
\end{figure}

\subsection{图论示例}

\begin{figure}[H]
    \centering
    \begin{tikzpicture}
        % 节点
        \node[mfnode] (A) at (0, 0) {A};
        \node[mfnode] (B) at (2, 1) {B};
        \node[mfnode] (C) at (2, -1) {C};
        \node[mfnode] (D) at (4, 0) {D};
        \node[mfnode, highlight node] (E) at (6, 0) {E};
        % 边
        \draw[graphedge] (A) -- node[edgelabel] {3} (B);
        \draw[graphedge] (A) -- node[edgelabel] {5} (C);
        \draw[graphedge] (B) -- node[edgelabel] {2} (D);
        \draw[graphedge] (C) -- node[edgelabel] {4} (D);
        \draw[graphedge, highlight edge] (D) -- node[edgelabel] {1} (E);
    \end{tikzpicture}
    \caption{加权图 (使用 mfnode + graphedge + edgelabel)}
\end{figure}

\subsection{流程图示例}

\begin{figure}[H]
    \centering
    \begin{tikzpicture}[node distance=12mm]
        \node[terminal] (start) {开始};
        \node[io, below=of start] (input) {输入 $n$};
        \node[process, below=of input] (init) {$sum \leftarrow 0$};
        \node[decision, below=of init] (cond) {$n > 0$?};
        \node[process, below=of cond] (calc) {$sum \leftarrow sum + n$};
        \node[process, below=of calc] (dec) {$n \leftarrow n - 1$};
        \node[io, right=25mm of cond] (output) {输出 $sum$};
        \node[terminal, below=of output] (end) {结束};
        
        % 连接
        \draw[flowedge] (start) -- (input);
        \draw[flowedge] (input) -- (init);
        \draw[flowedge] (init) -- (cond);
        \draw[flowedge] (cond) -- node[left] {是} (calc);
        \draw[flowedge] (calc) -- (dec);
        \draw[flowedge] (dec.west) -- ++(-1,0) |- (cond.west);
        \draw[flowedge] (cond) -- node[above] {否} (output);
        \draw[flowedge] (output) -- (end);
    \end{tikzpicture}
    \caption{求和算法流程图 (使用 process/decision/terminal/io)}
\end{figure}

\subsection{有限状态机}

\begin{figure}[H]
    \centering
    \begin{tikzpicture}[node distance=25mm]
        \node[initial state] (q0) {$q_0$};
        \node[state, right=of q0] (q1) {$q_1$};
        \node[final state, right=of q1] (q2) {$q_2$};
        
        \draw[transition] (q0) edge[bend left] node[above] {$a$} (q1);
        \draw[transition] (q1) edge[bend left] node[below] {$b$} (q0);
        \draw[transition] (q1) -- node[above] {$a$} (q2);
        \draw[transition] (q0) edge[loop above] node {$b$} (q0);
        \draw[transition] (q2) edge[loop above] node {$a,b$} (q2);
    \end{tikzpicture}
    \caption{有限自动机 (使用 state/initial state/final state)}
\end{figure}

\subsection{拓扑空间示意}

\begin{figure}[H]
    \centering
    \begin{tikzpicture}
        % 拓扑空间 X
        \draw[topology space] (0,0) ellipse (2.5cm and 1.8cm);
        \node at (0, 2.1) {$X$};
        
        % 开集
        \draw[open set] (-0.8, 0) ellipse (1cm and 0.8cm);
        \node at (-0.8, 1) {\small $U$ (开集)};
        
        % 点
        \node[point, label=below:$x$] at (-0.8, 0) {};
        \node[point, label=below:$y$] at (1.2, -0.3) {};
        
        % 映射箭头
        \draw[continuous map] (3.5, 0) -- node[above] {$f$} (5.5, 0);
        
        % 拓扑空间 Y
        \draw[topology space] (7.5, 0) ellipse (2cm and 1.5cm);
        \node at (7.5, 1.8) {$Y$};
        
        % 像
        \draw[closed set] (7.5, 0) circle (0.8cm);
        \node at (7.5, 1.1) {\small $f(U)$};
        \node[point, label=below:$f(x)$] at (7.5, 0) {};
    \end{tikzpicture}
    \caption{连续映射示意图 (使用 topology space/open set/closed set)}
\end{figure}

\begin{notice}{TikZ 样式使用说明}
    所有样式可直接使用或组合,例如:
    \begin{itemize}
        \item \verb|\node[mfnode] (A) at (0,0) {A};| - 创建圆形节点
        \item \verb|\draw[graphedge] (A) -- (B);| - 绘制无向边
        \item \verb|\node[state, highlight node] ...| - 组合样式
    \end{itemize}
\end{notice}

% ═══════════════════════════════════════════════════════════════════════════
\section{高级拓扑图形}
% ═══════════════════════════════════════════════════════════════════════════

MindFlow 提供了专业的拓扑学图形绘制命令,支持常见拓扑曲面的可视化。

\begin{table}[H]
    \centering
    \caption{拓扑绘图命令}
    \begin{tabular}{lll}
        \toprule
        \textbf{命令} & \textbf{曲面} & \textbf{参数} \\
        \midrule
        \verb|\sphere| & 球面 $S^2$ & 位置 + 半径 \\
        \verb|\torus| & 环面 $T^2$ & 位置 + 大/小半径 \\
        \verb|\cylinder| & 圆柱面 & 位置 + 半径 + 高度 \\
        \verb|\kleinbottle| & Klein 瓶 & 位置 + 缩放 \\
        \verb|\bouquet| & 八字形 & 位置 + 缩放 \\
        \verb|\fiberbundle| & 纤维丛 & 位置 + 底空间半径 + 纤维高度 \\
        \bottomrule
    \end{tabular}
\end{table}

\subsection{基本拓扑曲面}

\begin{figure}[H]
    \centering
    \begin{tikzpicture}
        % 球面
        \sphere{0}{0}{1.2}
        \node[below] at (0,-1.8) {球面 $S^2$};
        
        % 环面
        \torus{5}{0}{1.2}{0.4}
        \node[below] at (5,-1.2) {环面 $T^2$};
        
        % 圆柱
        \cylinder{10}{-1}{0.8}{2}
        \node[below] at (10,-1.5) {圆柱面};
    \end{tikzpicture}
    \caption{常见拓扑曲面:球面、环面、圆柱面}
\end{figure}

\subsection{Klein 瓶与八字形}

\begin{figure}[H]
    \centering
    \begin{tikzpicture}
        % Klein 瓶
        \kleinbottle{0}{0}{1.2}
        \node[below] at (0,-2) {Klein 瓶 (不可定向)};
        
        % 八字形
        \bouquet{6}{0}{1.5}
        \node[below] at (6,-2) {八字形 $S^1 \vee S^1$};
        \node[below] at (6,-2.5) {\small $\pi_1 = \mathbb{Z} * \mathbb{Z}$};
    \end{tikzpicture}
    \caption{非平凡拓扑空间示例}
\end{figure}

\subsection{纤维丛示意}

\begin{figure}[H]
    \centering
    \begin{tikzpicture}
        \fiberbundle{0}{0}{2}{2.5}
        \node at (5, 1.2) {纤维丛 $F \to E \xrightarrow{\pi} B$};
        \node[align=left] at (5, 0) {\small $B$: 底空间\\ \small $E$: 总空间\\ \small $F$: 纤维};
    \end{tikzpicture}
    \caption{纤维丛的几何结构}
\end{figure}

\subsection{同伦等价}

\begin{figure}[H]
    \centering
    \begin{tikzpicture}
        % 实心圆盘
        \draw[thick, mf@thm@blue, fill=mf@thm@blue!20] (0,0) circle (1.2);
        \node[point] at (0,0) {};
        \node[below] at (0,-1.5) {$D^2$ (圆盘)};
        
        % 同伦等价箭头
        \draw[homotopy arrow] (2,0) -- (4,0) node[midway, above] {$\simeq$};
        
        % 点
        \node[point] at (5,0) {};
        \node[below] at (5,-1.5) {$\{*\}$ (点)};
        
        % 说明文字
        \node at (2.5, -2.5) {\small 圆盘可缩为一点 (可缩空间)};
    \end{tikzpicture}
    \caption{同伦等价示意}
\end{figure}

\subsection{交换图 (范畴论)}

\begin{figure}[H]
    \centering
    \begin{tikzpicture}[cd node/.append style={font=\large}]
        \node[cd node] (A) at (0,2) {$A$};
        \node[cd node] (B) at (3,2) {$B$};
        \node[cd node] (C) at (0,0) {$C$};
        \node[cd node] (D) at (3,0) {$D$};
        
        \draw[cd arrow] (A) -- node[above] {$f$} (B);
        \draw[cd arrow] (A) -- node[left] {$g$} (C);
        \draw[cd arrow] (B) -- node[right] {$h$} (D);
        \draw[cd arrow] (C) -- node[below] {$k$} (D);
        \draw[cd exists] (A) -- node[above right] {$\exists !$} (D);
        
        \node at (1.5, 1) {\small 交换};
    \end{tikzpicture}
    \caption{交换图:$h \circ f = k \circ g$}
\end{figure}

\subsection{向量场}

\begin{figure}[H]
    \centering
    \begin{tikzpicture}
        % 网格和坐标系
        \drawaxes{-0.5}{5}{4}
        
        % 向量场
        \vectorfield{0.5}{0.5}{4}{3}{1}
        
        % 奇点标记
        \node[source, label=below:\small 源] at (1.5,2) {};
        \node[sink, label=below:\small 汇] at (3.5,1.5) {};
        \node[saddle point, label=below:\small 鞍点] at (2.5,3) {};
    \end{tikzpicture}
    \caption{向量场与奇点分类}
\end{figure}

\subsection{曲线积分区域}

\begin{figure}[H]
    \centering
    \begin{tikzpicture}
        \drawaxes{-0.5}{4.5}{3.5}
        
        % 积分区域
        \fill[integral region] 
            plot[domain=0.5:3.5, samples=50] (\x, {0.5*sin(deg(\x))+1.5})
            -- (3.5,0) -- (0.5,0) -- cycle;
        
        % 曲线
        \draw[function curve, very thick] 
            plot[domain=0.5:3.5, samples=50] (\x, {0.5*sin(deg(\x))+1.5});
        
        % 边界标记
        \draw[thick] (0.5,0) -- (0.5, 1.74);
        \draw[thick] (3.5,0) -- (3.5, 1.32);
        
        \node at (2, 0.8) {$\int_a^b f(x)\,\dd x$};
        \node[below] at (0.5,0) {\small $a$};
        \node[below] at (3.5,0) {\small $b$};
    \end{tikzpicture}
    \caption{定积分的几何意义}
\end{figure}

\begin{tip}{高级拓扑绘图建议}
    \begin{enumerate}
        \item 使用 \verb|\sphere|, \verb|\torus| 等命令快速绘制标准曲面
        \item 交换图使用 \verb|cd arrow|, \verb|cd mono|, \verb|cd epi| 等样式
        \item 向量场使用 \verb|\vectorfield| 命令自动生成
        \item 所有命令支持可选参数自定义 TikZ 样式
    \end{enumerate}
\end{tip}

% ═══════════════════════════════════════════════════════════════════════════
\section{古卷轴与公式注释}
% ═══════════════════════════════════════════════════════════════════════════

\subsection{古卷轴 (Papyrus) 效果}

使用 \verb|\papyrus| 命令创建类似古老羊皮纸卷轴效果的装饰文本框:

\begin{center}
\papyrus[1.2]{%
    \begin{minipage}{0.6\textwidth}
        \centering
        \textbf{欧拉恒等式}\\[0.5em]
        $e^{i\pi} + 1 = 0$\\[0.3em]
        \small 数学中最美丽的公式
    \end{minipage}
}
\end{center}

\begin{center}
\papyrus{%
    \begin{minipage}{0.7\textwidth}
        \textbf{麦克斯韦方程组}\\[0.3em]
        $\nabla \cdot \vec{E} = \frac{\rho}{\varepsilon_0}$, \quad
        $\nabla \cdot \vec{B} = 0$\\[0.2em]
        $\nabla \times \vec{E} = -\frac{\partial \vec{B}}{\partial t}$, \quad
        $\nabla \times \vec{B} = \mu_0 \vec{J} + \mu_0 \varepsilon_0 \frac{\partial \vec{E}}{\partial t}$
    \end{minipage}
}
\end{center}

\subsection{公式注释系统}

使用 \verb|formulaexplain| 环境为公式添加详细的分项解释:

\begin{formulaexplain}[2cm]
    % 公式组件
    \node[eq part=orange] (kinetic) {$-\frac{\hbar^2}{2m}\nabla^2$};
    \node[eq operator, right=0.3cm of kinetic] (plus1) {$+$};
    \node[eq part=red, right=0.3cm of plus1] (potential) {$V(\vec{r})$};
    \node[eq operator, right=0.5cm of potential] (psi) {$\psi = E\psi$};
    
    % 说明标签
    \node[eq label, above=of kinetic] (kinetic-label) {动能算符\\非相对论形式};
    \node[eq label, below=of potential] (potential-label) {势能项\\外场或相互作用};
    
    % 箭头
    \draw[eq arrow] (kinetic-label) -- (kinetic);
    \draw[eq arrow] (potential-label) -- (potential);
\end{formulaexplain}

\subsection{公式流程图 (Kohn-Sham 风格)}

使用 \verb|formulaflow| 环境创建自洽计算流程图:

\begin{formulaflow}[0.8cm]
    \node[ks input] (init) {初始密度猜测 $\rho_0(\vec{r})$};
    \node[ks box, below=of init] (ks) {构造 KS 方程 $\hat{H}_{KS}\phi_i = \varepsilon_i\phi_i$};
    \node[ks box, below=of ks] (solve) {求解单电子方程};
    \node[ks box, below=of solve] (density) {计算新密度 $\rho(\vec{r}) = \sum_i |\phi_i|^2$};
    \node[ks decision, below=of density] (conv) {收敛?};
    \node[ks output, below=1.5cm of conv] (output) {输出: 总能量 $E[\rho]$、电子结构};
    
    % 连接
    \draw[ks arrow] (init) -- (ks);
    \draw[ks arrow] (ks) -- (solve);
    \draw[ks arrow] (solve) -- (density);
    \draw[ks arrow] (density) -- (conv);
    \draw[ks arrow] (conv) -- node[right] {是} (output);
    \draw[ks arrow] (conv.west) -- ++(-2,0) |- node[pos=0.25, above] {否} (ks.west);
\end{formulaflow}

\begin{notice}{公式注释命令参考}
    \begin{itemize}
        \item \verb|\papyrus[缩放]{内容}| - 卷轴装饰框
        \item \verb|eq part=颜色| - 公式组件高亮框
        \item \verb|eq label| - 说明文字样式
        \item \verb|ks box|, \verb|ks input|, \verb|ks output| - 流程图框
    \end{itemize}
\end{notice}

% ═══════════════════════════════════════════════════════════════════════════
\section{计算机科学可视化}
% ═══════════════════════════════════════════════════════════════════════════

MindFlow 提供了专业的计算机科学图形样式,用于绘制类继承图、数据结构、算法流程等。

\begin{table}[H]
    \centering
    \caption{CS 可视化样式分类}
    \begin{tabular}{lll}
        \toprule
        \textbf{类别} & \textbf{主要样式} & \textbf{用途} \\
        \midrule
        层次结构 & \texttt{hasse node}, \texttt{inherit arrow} & 类继承/Hasse图 \\
        UML & \texttt{uml class}, \texttt{aggregation} & 类图设计 \\
        蝴蝶网络 & \texttt{bf node}, \texttt{bf op} & FFT/并行算法 \\
        数据结构 & \texttt{array cell}, \texttt{tree node} & 数组/树/链表 \\
        内存图 & \texttt{memory block}, \texttt{register} & 内存布局 \\
        状态机 & \texttt{fsm state}, \texttt{fsm trans} & FSM/DFA \\
        \bottomrule
    \end{tabular}
\end{table}

\subsection{类继承层次图}

\begin{hassediagram}[1.5cm]
    % 根类
    \node[abstract node] (object) {Object};
    
    % 第二层
    \node[abstract node, below left=of object, xshift=-1cm] (animal) {Animal};
    \node[interface node, below right=of object, xshift=1cm] (serializable) {$\ll$interface$\gg$\\Serializable};
    
    % 第三层
    \node[concrete node, below left=of animal] (dog) {Dog};
    \node[concrete node, below right=of animal] (cat) {Cat};
    
    % 连接
    \draw[inherit arrow] (animal) -- (object);
    \draw[implement arrow] (serializable) -- (object);
    \draw[inherit arrow] (dog) -- (animal);
    \draw[inherit arrow] (cat) -- (animal);
    \draw[implement arrow] (dog.east) to[bend left=20] (serializable.south);
\end{hassediagram}

\subsection{数据结构可视化}

\subsubsection{数组}

\begin{datastructure}
    \drawarray{0}{0}{5,2,8,1,9,3,7}
    \node[above] at (3cm, 0.6cm) {\textbf{int arr[7]}};
\end{datastructure}

\subsubsection{链表}

\begin{datastructure}
    \node[list node] (n1) at (0,0) {10 \nodepart{two} $\rightarrow$};
    \node[list node] (n2) at (2.5cm,0) {20 \nodepart{two} $\rightarrow$};
    \node[list node] (n3) at (5cm,0) {30 \nodepart{two} $\rightarrow$};
    \node[null node] (null) at (7cm,0) {NULL};
    
    \draw[list pointer] (n1.east) -- (n2.west);
    \draw[list pointer] (n2.east) -- (n3.west);
    \draw[list pointer] (n3.east) -- (null.west);
    
    \node[above] at (3cm, 0.7cm) {\textbf{单向链表}};
\end{datastructure}

\subsubsection{二叉树}

\begin{datastructure}[1.2cm]
    \node[tree node] (root) at (0,0) {8};
    \node[tree node] (l1) at (-1.5cm,-1.2cm) {4};
    \node[tree node] (r1) at (1.5cm,-1.2cm) {12};
    \node[tree node] (ll) at (-2.2cm,-2.4cm) {2};
    \node[tree node] (lr) at (-0.8cm,-2.4cm) {6};
    \node[tree node] (rl) at (0.8cm,-2.4cm) {10};
    \node[tree node] (rr) at (2.2cm,-2.4cm) {14};
    
    \draw[tree edge] (root) -- (l1);
    \draw[tree edge] (root) -- (r1);
    \draw[tree edge] (l1) -- (ll);
    \draw[tree edge] (l1) -- (lr);
    \draw[tree edge] (r1) -- (rl);
    \draw[tree edge] (r1) -- (rr);
    
    \node[above] at (0, 0.6cm) {\textbf{二叉搜索树}};
\end{datastructure}

\subsubsection{栈}

\begin{datastructure}
    \drawstack{0}{0}{return addr, local var, param 2, param 1}
    \node[above right] at (1.2cm, 2.5cm) {\textbf{调用栈}};
\end{datastructure}

\subsection{蝴蝶网络 (FFT)}

\begin{butterflynet}[0.8cm]
    % Stage 0
    \foreach \i/\v in {0/$a_0$, 1/$a_1$, 2/$a_2$, 3/$a_3$} {
        \node[bf node] (s0-\i) at (0, -\i*0.9cm) {\v};
    }
    
    % Stage 1
    \foreach \i in {0,1,2,3} {
        \node[bf node] (s1-\i) at (3cm, -\i*0.9cm) {};
    }
    
    % Stage 2
    \foreach \i/\v in {0/$A_0$, 1/$A_1$, 2/$A_2$, 3/$A_3$} {
        \node[bf node] (s2-\i) at (6cm, -\i*0.9cm) {\v};
    }
    
    % Stage 0-1 蝴蝶操作
    \node[bf op] (op01) at (1.5cm, -0.45cm) {$+$};
    \node[bf op] (op23) at (1.5cm, -2.25cm) {$+$};
    
    \draw[bf edge] (s0-0) -- (op01) -- (s1-0);
    \draw[bf edge] (s0-1) -- (op01) -- (s1-1);
    \draw[bf edge] (s0-2) -- (op23) -- (s1-2);
    \draw[bf edge] (s0-3) -- (op23) -- (s1-3);
    
    % Stage 1-2 蝴蝶操作 (交叉)
    \node[bf op] (op02) at (4.5cm, -0.9cm) {$+$};
    \node[bf op] (op13) at (4.5cm, -1.8cm) {$+$};
    
    \draw[bf edge] (s1-0) -- (op02) -- (s2-0);
    \draw[bf cross] (s1-2) -- (op02) -- (s2-2);
    \draw[bf edge] (s1-1) -- (op13) -- (s2-1);
    \draw[bf cross] (s1-3) -- (op13) -- (s2-3);
    
    % 阶段标签
    \node[below] at (0, -3.2cm) {\small Stage 0};
    \node[below] at (3cm, -3.2cm) {\small Stage 1};
    \node[below] at (6cm, -3.2cm) {\small Stage 2};
\end{butterflynet}

\subsection{有限状态机}

\begin{datastructure}[2.5cm]
    \node[fsm initial] (q0) at (0,0) {$q_0$};
    \node[fsm state] (q1) at (3cm,0) {$q_1$};
    \node[fsm state] (q2) at (6cm,0) {$q_2$};
    \node[fsm final] (q3) at (9cm,0) {$q_3$};
    
    \draw[fsm trans] (q0) -- node[fsm label, above] {0} (q1);
    \draw[fsm trans] (q1) -- node[fsm label, above] {1} (q2);
    \draw[fsm trans] (q2) -- node[fsm label, above] {0} (q3);
    \draw[fsm trans] (q1) edge[loop above] node[fsm label] {0} (q1);
    \draw[fsm trans] (q2) edge[bend left=40] node[fsm label, below] {1} (q0);
\end{datastructure}

\begin{tip}{CS 可视化使用建议}
    \begin{enumerate}
        \item 使用 \verb|hassediagram| 环境绘制类继承图
        \item 使用 \verb|datastructure| 环境绘制数据结构
        \item \verb|\drawarray| 和 \verb|\drawstack| 是便捷的快捷命令
        \item 蝴蝶网络使用 \verb|bf node| + \verb|bf op| 组合
    \end{enumerate}
\end{tip}

% ═══════════════════════════════════════════════════════════════════════════
\section{化学与物理可视化}
% ═══════════════════════════════════════════════════════════════════════════

MindFlow 提供了专业的化学和物理图形样式,用于绘制元素周期表、分子结构、能级图等。

\begin{table}[H]
    \centering
    \caption{化学/物理可视化样式分类}
    \begin{tabular}{lll}
        \toprule
        \textbf{类别} & \textbf{主要样式} & \textbf{用途} \\
        \midrule
        元素周期表 & \texttt{alkali metal}, \texttt{halogen} & 元素分类 \\
        粒子物理 & \texttt{quark}, \texttt{lepton}, \texttt{gauge boson} & 标准模型 \\
        分子结构 & \texttt{carbon}, \texttt{single bond} & 有机化学 \\
        能级图 & \texttt{energy level}, \texttt{transition arrow} & 量子力学 \\
        化学反应 & \texttt{reactant}, \texttt{reaction arrow} & 反应方程 \\
        \bottomrule
    \end{tabular}
\end{table}

\subsection{元素周期表}

元素周期表使用不同颜色区分元素类别:

\begin{figure}[H]
    \centering
    \begin{tikzpicture}[scale=0.6, every node/.style={font=\sffamily\small}]
        % 第一周期
        \node[nonmetal] at (0,0) {
            \begin{minipage}{1.5cm}\centering
                {\tiny 1}\\\textbf{\Large H}\\{\tiny Hydrogen}
            \end{minipage}
        };
        \node[noble gas] at (10cm,0) {
            \begin{minipage}{1.5cm}\centering
                {\tiny 2}\\\textbf{\Large He}\\{\tiny Helium}
            \end{minipage}
        };
        
        % 第二周期部分元素
        \node[alkali metal] at (0,-2cm) {
            \begin{minipage}{1.5cm}\centering
                {\tiny 3}\\\textbf{\Large Li}\\{\tiny Lithium}
            \end{minipage}
        };
        \node[alkaline earth] at (2cm,-2cm) {
            \begin{minipage}{1.5cm}\centering
                {\tiny 4}\\\textbf{\Large Be}\\{\tiny Beryllium}
            \end{minipage}
        };
        \node[nonmetal] at (6cm,-2cm) {
            \begin{minipage}{1.5cm}\centering
                {\tiny 6}\\\textbf{\Large C}\\{\tiny Carbon}
            \end{minipage}
        };
        \node[halogen] at (8cm,-2cm) {
            \begin{minipage}{1.5cm}\centering
                {\tiny 9}\\\textbf{\Large F}\\{\tiny Fluorine}
            \end{minipage}
        };
        \node[noble gas] at (10cm,-2cm) {
            \begin{minipage}{1.5cm}\centering
                {\tiny 10}\\\textbf{\Large Ne}\\{\tiny Neon}
            \end{minipage}
        };
    \end{tikzpicture}
    \caption{元素周期表片段 (使用 element 样式)}
\end{figure}

\begin{notice}{元素类别样式}
    \begin{itemize}
        \item \texttt{alkali metal} - 碱金属 (红色)
        \item \texttt{alkaline earth} - 碱土金属 (橙色)
        \item \texttt{nonmetal} - 非金属 (绿色)
        \item \texttt{halogen} - 卤素 (黄色)
        \item \texttt{noble gas} - 稀有气体 (青色)
    \end{itemize}
\end{notice}

\subsection{分子结构}

\subsubsection{水分子}

\begin{molecule}[1.5cm]
    \node[oxygen] (O) at (0,0) {O};
    \node[hydrogen] (H1) at (-1.2cm,-0.8cm) {H};
    \node[hydrogen] (H2) at (1.2cm,-0.8cm) {H};
    
    \draw[single bond] (O) -- (H1);
    \draw[single bond] (O) -- (H2);
    
    \node[above=0.3cm of O] {$\mathrm{H_2O}$};
\end{molecule}

\subsubsection{甲烷分子}

\begin{molecule}[1.2cm]
    \node[carbon] (C) at (0,0) {C};
    \node[hydrogen] (H1) at (0,1.2cm) {H};
    \node[hydrogen] (H2) at (0,-1.2cm) {H};
    \node[hydrogen] (H3) at (-1.2cm,0) {H};
    \node[hydrogen] (H4) at (1.2cm,0) {H};
    
    \draw[single bond] (C) -- (H1);
    \draw[single bond] (C) -- (H2);
    \draw[single bond] (C) -- (H3);
    \draw[single bond] (C) -- (H4);
    
    \node[below=1.5cm of C] {$\mathrm{CH_4}$ 甲烷};
\end{molecule}

\subsubsection{乙烯 (双键)}

\begin{molecule}[2cm]
    \node[carbon] (C1) at (0,0) {C};
    \node[carbon] (C2) at (2cm,0) {C};
    \node[hydrogen] (H1) at (-0.8cm,0.8cm) {H};
    \node[hydrogen] (H2) at (-0.8cm,-0.8cm) {H};
    \node[hydrogen] (H3) at (2.8cm,0.8cm) {H};
    \node[hydrogen] (H4) at (2.8cm,-0.8cm) {H};
    
    \draw[double bond] (C1) -- (C2);
    \draw[single bond] (C1) -- (H1);
    \draw[single bond] (C1) -- (H2);
    \draw[single bond] (C2) -- (H3);
    \draw[single bond] (C2) -- (H4);
    
    \node[below=1.2cm of C1, xshift=1cm] {$\mathrm{C_2H_4}$ 乙烯};
\end{molecule}

\subsection{能级图}

\begin{energydiagram}
    % 能级线
    \draw[ground state] (0,0) -- (2cm,0) node[energy label] {$n=1$ (基态)};
    \draw[energy level] (0,1.5cm) -- (2cm,1.5cm) node[energy label] {$n=2$};
    \draw[energy level] (0,2.5cm) -- (2cm,2.5cm) node[energy label] {$n=3$};
    \draw[excited state] (0,3.2cm) -- (2cm,3.2cm) node[energy label] {$n=4$};
    
    % 跃迁箭头
    \draw[transition arrow] (1cm,0.1cm) -- (1cm,1.4cm);
    \node[right, font=\footnotesize] at (1.2cm,0.75cm) {吸收};
    
    \draw[transition arrow] (0.5cm,2.4cm) -- (0.5cm,0.1cm);
    \node[left, font=\footnotesize] at (0.3cm,1.25cm) {发射};
    
    % 能量轴
    \draw[->, thick] (-0.5cm,-0.3cm) -- (-0.5cm,3.5cm) node[above] {$E$};
\end{energydiagram}

\subsection{化学反应图}

\begin{datastructure}[2cm]
    \node[reactant] (r1) at (0,0) {$\mathrm{2H_2}$};
    \node[reactant] (r2) at (2cm,0) {$\mathrm{O_2}$};
    \node at (1cm,0) {$+$};
    \node[product] (p1) at (6cm,0) {$\mathrm{2H_2O}$};
    
    \draw[reaction arrow] (3cm,0) -- (4.8cm,0) 
        node[catalyst] {点燃}
        node[condition] {高温};
\end{datastructure}

\begin{tip}{化学可视化使用建议}
    \begin{enumerate}
        \item 使用 \verb|periodictable| 环境绘制元素周期表
        \item 使用 \verb|molecule| 环境绘制分子结构
        \item 原子样式: \verb|carbon|, \verb|hydrogen|, \verb|oxygen|, \verb|nitrogen|
        \item 键样式: \verb|single bond|, \verb|double bond|, \verb|triple bond|
    \end{enumerate}
\end{tip}

\subsection{粒子物理标准模型}

\begin{figure}[H]
    \centering
    \begin{tikzpicture}[scale=0.9, every node/.style={font=\sffamily}]
        % 夸克
        \node[quark] (u) at (0,3) {
            \begin{minipage}{1.4cm}\centering
                {\tiny 2.2 MeV}\\\textbf{u}\\{\tiny up}
            \end{minipage}
        };
        \node[quark] (d) at (0,1.5) {
            \begin{minipage}{1.4cm}\centering
                {\tiny 4.7 MeV}\\\textbf{d}\\{\tiny down}
            \end{minipage}
        };
        \node[quark] (c) at (1.8cm,3) {
            \begin{minipage}{1.4cm}\centering
                {\tiny 1.3 GeV}\\\textbf{c}\\{\tiny charm}
            \end{minipage}
        };
        \node[quark] (s) at (1.8cm,1.5) {
            \begin{minipage}{1.4cm}\centering
                {\tiny 96 MeV}\\\textbf{s}\\{\tiny strange}
            \end{minipage}
        };
        
        % 轻子
        \node[lepton] (e) at (0,0) {
            \begin{minipage}{1.4cm}\centering
                {\tiny 0.511 MeV}\\\textbf{e}\\{\tiny electron}
            \end{minipage}
        };
        \node[lepton] (nu) at (1.8cm,0) {
            \begin{minipage}{1.4cm}\centering
                {\tiny $<$1 eV}\\$\boldsymbol{\nu_e}$\\{\tiny neutrino}
            \end{minipage}
        };
        
        % 玻色子
        \node[gauge boson] (g) at (4.5cm,3) {
            \begin{minipage}{1.4cm}\centering
                {\tiny 0}\\\textbf{g}\\{\tiny gluon}
            \end{minipage}
        };
        \node[gauge boson] (gamma) at (4.5cm,1.5) {
            \begin{minipage}{1.4cm}\centering
                {\tiny 0}\\$\boldsymbol{\gamma}$\\{\tiny photon}
            \end{minipage}
        };
        \node[gauge boson] (W) at (4.5cm,0) {
            \begin{minipage}{1.4cm}\centering
                {\tiny 80.4 GeV}\\\textbf{W}\\{\tiny W boson}
            \end{minipage}
        };
        \node[scalar boson] (H) at (6.3cm,1.5) {
            \begin{minipage}{1.4cm}\centering
                {\tiny 125 GeV}\\\textbf{H}\\{\tiny Higgs}
            \end{minipage}
        };
        
        % 标签
        \node[above, font=\small\bfseries] at (0.9cm,3.8) {QUARKS};
        \node[above, font=\small\bfseries] at (0.9cm,0.8) {LEPTONS};
        \node[above, font=\small\bfseries] at (5.4cm,3.8) {BOSONS};
    \end{tikzpicture}
    \caption{粒子物理标准模型简化图}
\end{figure}

\subsection{几何分类图}

\begin{figure}[H]
    \centering
    \begin{tikzpicture}[node distance=1.5cm, >=Stealth]
        % 几何形状节点
        \node[hasse node] (quad) at (0,4) {四边形};
        
        \node[hasse node] (trap) at (-3,2.5) {梯形};
        \node[hasse node] (para) at (0,2.5) {平行四边形};
        \node[hasse node] (kite) at (3,2.5) {风筝形};
        
        \node[hasse node] (rect) at (-1.5,1) {矩形};
        \node[hasse node] (rhom) at (1.5,1) {菱形};
        
        \node[concrete node] (squa) at (0,-0.5) {正方形};
        
        % 连接线
        \draw[hasse edge] (quad) -- (trap);
        \draw[hasse edge] (quad) -- (para);
        \draw[hasse edge] (quad) -- (kite);
        \draw[hasse edge] (para) -- (rect);
        \draw[hasse edge] (para) -- (rhom);
        \draw[hasse edge] (kite) -- (rhom);
        \draw[hasse edge] (rect) -- (squa);
        \draw[hasse edge] (rhom) -- (squa);
    \end{tikzpicture}
    \caption{四边形分类 Hasse 图}
\end{figure}

% ═══════════════════════════════════════════════════════════════════════════
\section{高级数学流程图}
% ═══════════════════════════════════════════════════════════════════════════

MindFlow 提供了专业的科学计算和研究工作流流程图样式。

\begin{table}[H]
    \centering
    \caption{高级流程图样式分类}
    \begin{tabular}{lll}
        \toprule
        \textbf{类别} & \textbf{主要样式} & \textbf{用途} \\
        \midrule
        分组框 & \texttt{basic box}, \texttt{loop box} & DMFT循环 \\
        科学计算 & \texttt{sci input}, \texttt{sci step} & 数值方法 \\
        迭代算法 & \texttt{iter init}, \texttt{iter converge} & Newton迭代 \\
        研究工作流 & \texttt{phase box}, \texttt{milestone} & 项目管理 \\
        \bottomrule
    \end{tabular}
\end{table}

\subsection{迭代算法流程图 (Newton 法)}

\begin{iterflow}[1.5cm]
    \node[iter init] (init) {初始化 $x_0$, 设置 $\varepsilon$};
    \node[iter step, below=of init] (compute) {计算 $f(x_n)$ 和 $f'(x_n)$};
    \node[iter step, below=of compute] (update) {更新 $x_{n+1} = x_n - \dfrac{f(x_n)}{f'(x_n)}$};
    \node[iter converge, below=of update] (check) {$|x_{n+1} - x_n| < \varepsilon$?};
    \node[iter result, below=of check] (result) {输出根 $x^* = x_{n+1}$};
    
    % 箭头
    \draw[sci arrow] (init) -- (compute);
    \draw[sci arrow] (compute) -- (update);
    \draw[sci arrow] (update) -- (check);
    \draw[sci arrow] (check) -- node[right] {是} (result);
    \draw[iter loop] (check.west) -- ++(-1.5cm,0) |- node[pos=0.25, left] {否} (compute.west);
\end{iterflow}

\subsection{科学计算流程}

\begin{sciflow}[1.2cm]
    % 输入
    \node[sci input] (input) {读取初始条件};
    
    % 主计算循环
    \node[sci step, below=of input] (mesh) {生成网格};
    \node[sci step, below=of mesh] (assemble) {组装刚度矩阵};
    \node[sci step, below=of assemble] (solve) {求解线性系统 $Ax = b$};
    \node[sci decision, below=of solve] (converge) {收敛?};
    
    % 输出
    \node[sci output, below=of converge] (output) {输出结果};
    
    % 连接
    \draw[sci arrow] (input) -- (mesh);
    \draw[sci arrow] (mesh) -- (assemble);
    \draw[sci arrow] (assemble) -- (solve);
    \draw[sci arrow] (solve) -- (converge);
    \draw[sci arrow] (converge) -- node[right] {是} (output);
    \draw[sci arrow] (converge.east) -- ++(1cm,0) |- node[pos=0.25, right] {否,细化} (mesh.east);
\end{sciflow}

\subsection{研究工作流}

\begin{workflow}[0.03cm]
    \node[phase box=flow@green] (plan) {问题定义};
    \node[phase box=flow@blue, right=of plan] (model) {模型构建};
    \node[phase box=flow@purple, right=of model] (compute) {数值求解};
    \node[phase box=flow@orange, right=of compute] (analyze) {结果分析};
    \node[phase box=flow@red, right=of analyze] (publish) {论文发表};
    
    \draw[phase arrow] (plan) -- (model);
    \draw[phase arrow] (model) -- (compute);
    \draw[phase arrow] (compute) -- (analyze);
    \draw[phase arrow] (analyze) -- (publish);
    
    % 反馈
    \draw[feedback loop] (analyze) to[bend left=40] node[above, font=\small] {修正} (model);
\end{workflow}

\begin{tip}{高级流程图使用建议}
    \begin{enumerate}
        \item 使用 \verb|iterflow| 环境绘制迭代算法
        \item 使用 \verb|sciflow| 环境绘制科学计算流程
        \item 使用 \verb|workflow| 环境绘制研究工作流
        \item 样式可通过 \verb|basic box=颜色| 自定义
    \end{enumerate}
\end{tip}

% ═══════════════════════════════════════════════════════════════════════════
\section{3D 几何可视化}
% ═══════════════════════════════════════════════════════════════════════════

MindFlow 提供了 3D 多面体和几何图形的绘制样式,需要 \texttt{tikz-3dplot} 包支持。

\begin{table}[H]
    \centering
    \caption{3D 几何样式分类}
    \begin{tabular}{lll}
        \toprule
        \textbf{类别} & \textbf{主要样式} & \textbf{用途} \\
        \midrule
        多面体 & \texttt{poly face}, \texttt{poly edge} & 面/边绘制 \\
        曲面 & \texttt{surface grid}, \texttt{skew edge} & 参数曲面 \\
        坐标系 & \texttt{3d axis} & 3D坐标轴 \\
        晶体 & \texttt{rhombic face}, \texttt{lattice point} & 晶体结构 \\
        \bottomrule
    \end{tabular}
\end{table}

\subsection{长方体 (隐藏边处理)}

\begin{figure}[H]
    \centering
    \tdplotsetmaincoords{70}{110}
    \begin{tikzpicture}[tdplot_main_coords, scale=1.2]
        \pgfmathsetmacro{\a}{2.5}
        \pgfmathsetmacro{\b}{2.0}
        \pgfmathsetmacro{\c}{1.5}
        
        % 底面 (部分隐藏)
        \draw[poly edge hidden] (0,0,0) -- (\a,0,0);
        \draw[poly edge hidden] (0,0,0) -- (0,\b,0);
        \draw[poly edge] (\a,0,0) -- (\a,\b,0) -- (0,\b,0);
        
        % 顶面
        \draw[poly edge] (0,0,\c) -- (\a,0,\c) -- (\a,\b,\c) -- (0,\b,\c) -- cycle;
        
        % 垂直边
        \draw[poly edge hidden] (0,0,0) -- (0,0,\c);
        \draw[poly edge] (\a,0,0) -- (\a,0,\c);
        \draw[poly edge] (\a,\b,0) -- (\a,\b,\c);
        \draw[poly edge] (0,\b,0) -- (0,\b,\c);
        
        % 坐标轴
        \draw[3d axis] (0,0,0) -- (3.5,0,0) node[right] {$x$};
        \draw[3d axis] (0,0,0) -- (0,3,0) node[above] {$y$};
        \draw[3d axis] (0,0,0) -- (0,0,2.5) node[above] {$z$};
    \end{tikzpicture}
    \caption{长方体与3D坐标系}
\end{figure}

\subsection{正四面体}

\begin{figure}[H]
    \centering
    \tdplotsetmaincoords{70}{120}
    \begin{tikzpicture}[tdplot_main_coords, scale=2]
        % 顶点坐标
        \coordinate (A) at (0,0,0);
        \coordinate (B) at (1,0,0);
        \coordinate (C) at (0.5,0.866,0);
        \coordinate (D) at (0.5,0.289,0.816);
        
        % 底面
        \draw[poly face] (A) -- (B) -- (C) -- cycle;
        
        % 侧面
        \draw[poly face] (A) -- (B) -- (D) -- cycle;
        \draw[poly face] (B) -- (C) -- (D) -- cycle;
        \draw[poly face] (C) -- (A) -- (D) -- cycle;
        
        % 顶点标记
        \node[poly vertex] at (A) {};
        \node[poly vertex] at (B) {};
        \node[poly vertex] at (C) {};
        \node[poly vertex] at (D) {};
        
        % 标签
        \node[vertex label, below left] at (A) {$A$};
        \node[vertex label, below right] at (B) {$B$};
        \node[vertex label, right] at (C) {$C$};
        \node[vertex label, above] at (D) {$D$};
    \end{tikzpicture}
    \caption{正四面体 (使用 \texttt{poly face} 样式)}
\end{figure}

\begin{notice}{3D 几何环境}
    \begin{itemize}
        \item \verb|geometry3d[仰角]{方位角}| - 通用 3D 环境
        \item \verb|polyhedron[仰角]{方位角}| - 多面体专用
        \item 需要 \verb|\usepackage{tikz-3dplot}| (已在 cls 中加载)
    \end{itemize}
\end{notice}

% ═══════════════════════════════════════════════════════════════════════════
\section{神经网络可视化}
% ═══════════════════════════════════════════════════════════════════════════

MindFlow 提供了专业的神经网络架构可视化样式。

\begin{table}[H]
    \centering
    \caption{神经网络样式分类}
    \begin{tabular}{lll}
        \toprule
        \textbf{类别} & \textbf{主要样式} & \textbf{用途} \\
        \midrule
        神经元 & \texttt{input/hidden/output neuron} & 不同层节点 \\
        连接 & \texttt{nn edge}, \texttt{skip connection} & 全连接/残差 \\
        层 & \texttt{conv layer}, \texttt{pool layer} & CNN组件 \\
        VAE & \texttt{mu block}, \texttt{sigma block} & 变分自编码器 \\
        \bottomrule
    \end{tabular}
\end{table}

\subsection{多层感知机 (MLP)}

\begin{neuralnet}
    % 输入层
    \foreach \i in {1,2,3} {
        \node[input neuron] (i-\i) at (0, 2-\i) {};
    }
    \node[layer label, above=0.3cm of i-1] {输入层};
    
    % 隐藏层1
    \foreach \i in {1,2,3,4} {
        \node[hidden neuron] (h1-\i) at (2.5, 2.5-\i) {};
    }
    \node[layer label, above=0.3cm of h1-1] {隐藏层};
    
    % 输出层
    \foreach \i in {1,2} {
        \node[output neuron] (o-\i) at (5, 1.5-\i) {};
    }
    \node[layer label, above=0.3cm of o-1] {输出层};
    
    % 全连接
    \foreach \i in {1,2,3} {
        \foreach \j in {1,2,3,4} {
            \draw[nn edge] (i-\i) -- (h1-\j);
        }
    }
    \foreach \i in {1,2,3,4} {
        \foreach \j in {1,2} {
            \draw[nn edge] (h1-\i) -- (o-\j);
        }
    }
    
    % 输入标签
    \node[left=0.3cm of i-1] {$x_1$};
    \node[left=0.3cm of i-2] {$x_2$};
    \node[left=0.3cm of i-3] {$x_3$};
    
    % 输出标签
    \node[right=0.3cm of o-1] {$\hat{y}_1$};
    \node[right=0.3cm of o-2] {$\hat{y}_2$};
\end{neuralnet}

\subsection{变分自编码器 (VAE)}

\begin{autoencoder}
    % Encoder
    \node[fc layer, minimum height=3cm] (enc1) at (0,0) {};
    \node[fc layer, minimum height=2cm] (enc2) at (1.5,0) {};
    
    % Latent space
    \node[mu block] (mu) at (4, 0.8) {$\mu$};
    \node[sigma block] (sigma) at (4, -0.8) {$\sigma$};
    \node[sample block] (z) at (6, 0) {$z$};
    
    % Decoder
    \node[fc layer, minimum height=2cm] (dec1) at (8,0) {};
    \node[fc layer, minimum height=3cm] (dec2) at (9.5,0) {};
    
    % 连接
    \draw[nn edge, thick] (enc1) -- (enc2);
    \draw[nn edge, thick] (enc2) -- (mu);
    \draw[nn edge, thick] (enc2) -- (sigma);
    \draw[nn edge, thick] (mu) -- (z);
    \draw[nn edge, thick] (sigma) -- (z);
    \draw[nn edge, thick] (z) -- (dec1);
    \draw[nn edge, thick] (dec1) -- (dec2);
    
    % 标签
    \node[layer label, below=0.5cm of enc1] {Encoder};
    \node[layer label, below=0.5cm of z] {Latent};
    \node[layer label, below=0.5cm of dec2] {Decoder};
    
    % 输入输出
    \node[left=0.3cm of enc1] {$x$};
    \node[right=0.3cm of dec2] {$\hat{x}$};
\end{autoencoder}

\begin{tip}{神经网络可视化使用建议}
    \begin{enumerate}
        \item 使用 \verb|neuralnet| 环境绘制 MLP
        \item 使用 \verb|autoencoder| 环境绘制 VAE/AE
        \item 使用 \verb|convnet| 环境绘制 CNN
        \item 快捷命令: \verb|\drawlayer|, \verb|\connectlayers|
    \end{enumerate}
\end{tip}

% ═══════════════════════════════════════════════════════════════════════════
\section{概率树与卷积运算}
% ═══════════════════════════════════════════════════════════════════════════

MindFlow 提供了概率树图和卷积运算可视化样式。

\subsection{概率树}

\begin{figure}[H]
    \centering
    \begin{tikzpicture}[
        grow=right,
        level distance=2.5cm,
        sibling distance=12mm,
        edge from parent/.style={prob edge},
        >=Stealth,
        level 1/.style={sibling distance=25mm},
        level 2/.style={sibling distance=12mm}
    ]
        \node[prob root] {}
            child {
                node[prob event] {$A$}
                child {
                    node[prob outcome] {$A \cap B$}
                    edge from parent node[prob label, above] {0.6}
                }
                child {
                    node[prob outcome] {$A \cap \bar{B}$}
                    edge from parent node[prob label, below] {0.4}
                }
                edge from parent node[prob label highlight, above] {0.3}
            }
            child {
                node[prob event highlight] {$\bar{A}$}
                child {
                    node[prob outcome] {$\bar{A} \cap B$}
                    edge from parent node[prob label, above] {0.2}
                }
                child {
                    node[prob outcome] {$\bar{A} \cap \bar{B}$}
                    edge from parent node[prob label, below] {0.8}
                }
                edge from parent[prob edge highlight] node[prob label highlight, below] {0.7}
            };
    \end{tikzpicture}
    \caption{条件概率树}
\end{figure}

\subsection{卷积运算}

\begin{figure}[H]
    \centering
    \begin{tikzpicture}[node distance=0.5cm]
        % 输入矩阵
        \matrix (I) [input matrix] {
            1 & 0 & |[conv input highlight]| 1 & |[conv input highlight]| 0 \\
            0 & 1 & |[conv input highlight]| 1 & |[conv input highlight]| 1 \\
            1 & 1 & 0 & 0 \\
            0 & 1 & 0 & 1 \\
        };
        \node[below=0.3cm of I] {输入 $\mathbf{I}$};
        
        % 卷积符号
        \node[conv operator, right=0.8cm of I] (star) {$*$};
        
        % 卷积核
        \matrix (K) [kernel matrix, right=0.8cm of star] {
            1 & 0 \\
            0 & 1 \\
        };
        \node[below=0.3cm of K] {核 $\mathbf{K}$};
        
        % 等号
        \node[conv operator, right=0.8cm of K] (eq) {$=$};
        
        % 输出矩阵
        \matrix (O) [output matrix, right=0.8cm of eq] {
            2 & |[conv output highlight]| 2 & 1 \\
            2 & 1 & 1 \\
            2 & 1 & 1 \\
        };
        \node[below=0.3cm of O] {输出 $\mathbf{I * K}$};
        
        % 映射线
        \draw[conv map] (I-1-4.north east) -- (K-1-1.north west);
        \draw[conv map] (I-2-4.south east) -- (K-2-1.south west);
        \draw[output map] (K-1-2.north east) -- (O-1-2.north west);
        \draw[output map] (K-2-2.south east) -- (O-1-2.south west);
    \end{tikzpicture}
    \caption{2D 卷积运算示意图}
\end{figure}

\begin{notice}{概率/卷积样式}
    \begin{itemize}
        \item \texttt{prob event}, \texttt{prob edge} - 概率树节点/边
        \item \texttt{prob label}, \texttt{prob label highlight} - 概率标签
        \item \texttt{input matrix}, \texttt{kernel matrix} - 卷积矩阵
        \item \texttt{conv input highlight}, \texttt{conv output highlight} - 区域高亮
    \end{itemize}
\end{notice}

% ═══════════════════════════════════════════════════════════════════════════
\section{TQFT 协边图}
% ═══════════════════════════════════════════════════════════════════════════

MindFlow 支持拓扑量子场论 (TQFT) 的协边图绘制,使用 \texttt{tqft} TikZ 库。

\subsection{基础协边}

\begin{figure}[H]
    \centering
    \begin{tikzpicture}[
        every tqft/.append style={
            transform shape,
            rotate=90,
            tqft/circle x radius=7pt,
            tqft/boundary separation=1cm,
            tqft/view from=incoming
        }
    ]
        % Pair of pants (乘法)
        \pic[
            tqft/pair of pants,
            name=pants,
            every incoming lower boundary component/.style={draw},
            every outgoing lower boundary component/.style={draw},
            cobordism edge/.style={draw},
        ];
        \node[below=0.5cm of pants-incoming boundary] {$\mu: V \otimes V \to V$};
        \node[above=0.3cm of pants-outgoing boundary 1, xshift=-0.3cm] {\small Pair of Pants};
    \end{tikzpicture}
    \hspace{2cm}
    \begin{tikzpicture}[
        every tqft/.append style={
            transform shape,
            rotate=90,
            tqft/circle x radius=7pt,
            tqft/boundary separation=1cm,
            tqft/view from=incoming
        }
    ]
        % Cup (迹)
        \pic[
            tqft/cup,
            name=cup,
            cobordism edge/.style={draw},
        ];
        \node[below=0.5cm of cup-incoming boundary] {$\mathrm{tr}: V \to \mathbb{C}$};
        \node[above=0.5cm of cup-incoming boundary] {\small Cup};
    \end{tikzpicture}
    \caption{TQFT 基本协边: Pair of Pants (乘法) 和 Cup (迹)}
\end{figure}

\subsection{柱面 (恒等态射)}

\begin{figure}[H]
    \centering
    \begin{tikzpicture}[
        every tqft/.append style={
            transform shape,
            rotate=90,
            tqft/circle x radius=7pt,
            tqft/boundary separation=1cm,
            tqft/view from=incoming
        }
    ]
        \pic[
            tqft/cylinder,
            name=cyl,
            every incoming lower boundary component/.style={draw},
            every outgoing lower boundary component/.style={draw},
            cobordism edge/.style={draw},
        ];
        
        \node[below=0.5cm of cyl-incoming boundary] {$\mathrm{id}: V \to V$};
        
        % 注释
        \draw[space param] ($(cyl-incoming boundary.west) - (0.2,0)$) 
            node[below] {$\sigma$} to[bend left=40] ++(0,0.5);
    \end{tikzpicture}
    \caption{柱面协边 (恒等态射)}
\end{figure}

\begin{notice}{TQFT 样式}
    \begin{itemize}
        \item \texttt{tqft/pair of pants} - 乘法协边
        \item \texttt{tqft/cup}, \texttt{tqft/cap} - 迹/余迹
        \item \texttt{tqft/cylinder} - 恒等态射
        \item 环境: \texttt{cobordism}, \texttt{tqftdiagram}
    \end{itemize}
\end{notice}

\subsection{晶格与拓扑等价}

\begin{figure}[H]
    \centering
    \begin{tikzpicture}
        % 晶格网格
        \draw[lattice grid] (-2,-2) grid (2,2);
        
        % 坐标轴
        \draw[lattice axis] (-2.3,0) -- (2.3,0);
        \draw[lattice axis] (0,-2.3) -- (0,2.3);
        
        % 基本域
        \node[fundamental domain, label=below:$2\pi$, label=right:$2\pi$] at (0.5,0.5) {};
    \end{tikzpicture}
    \hspace{1.5cm}
    \begin{tikzpicture}
        % 方形 -> 环面
        \draw[identification arrow] (0,0) -- (2,0);
        \draw[identification arrow reversed] (0,2) -- (2,2);
        \draw[identification arrow] (0,0) -- (0,2);
        \draw[identification arrow reversed] (2,0) -- (2,2);
        
        \node at (1,-0.5) {$a$};
        \node at (1,2.5) {$a$};
        \node at (-0.5,1) {$b$};
        \node at (2.5,1) {$b$};
        
        \node at (1,-1.2) {$\mathbb{T}^2 = \mathbb{R}^2/\mathbb{Z}^2$};
    \end{tikzpicture}
    \caption{晶格基本域与环面粘合}
\end{figure}

\begin{notice}{晶格/拓扑样式}
    \begin{itemize}
        \item \texttt{lattice grid}, \texttt{lattice axis} - 晶格网格/轴
        \item \texttt{fundamental domain} - 基本域
        \item \texttt{identification arrow} - 等价粘合箭头
        \item 环境: \texttt{latticediag}, \texttt{surfaceconstruct}
    \end{itemize}
\end{notice}

% ═══════════════════════════════════════════════════════════════════════════
\section{电磁波与光学}
% ═══════════════════════════════════════════════════════════════════════════

MindFlow 提供电磁波和光学可视化样式,包括偏振光等。

\subsection{线偏振波}

\begin{figure}[H]
    \centering
    \begin{tikzpicture}[>=Stealth]
        % 传播方向
        \draw[propagation arrow] (0,0) -- (7,0) node[right] {$z$};
        
        % E场波形
        \draw[wave curve, domain=0:6.5, variable=\x] 
            plot (\x, {0.8*sin(deg(\x*1.5))});
        
        % E场向量
        \foreach \x in {0.5, 1.5, 2.5, 3.5, 4.5, 5.5} {
            \pgfmathsetmacro{\y}{0.8*sin(deg(\x*1.5))}
            \draw[E field] (\x, 0) -- (\x, \y);
        }
        
        % 标签
        \node[em@efield] at (3.5, 1.2) {$\vec{E}$};
        \node at (3.5, -0.8) {线偏振波};
    \end{tikzpicture}
    \caption{线偏振电磁波 (E 场)}
\end{figure}

\subsection{偏振片}

\begin{figure}[H]
    \centering
    \begin{tikzpicture}[>=Stealth]
        % 入射光
        \draw[propagation arrow] (-3,0) -- (-0.5,0);
        \node at (-1.75, 0.5) {非偏振光};
        
        % 偏振片
        \draw[polarizer] (0,-1.2) rectangle (0.3,1.2);
        \draw[polarization axis] (0.15,-1.5) -- (0.15,1.5);
        \node at (0.15, -1.8) {偏振片};
        
        % 透射光
        \draw[propagation arrow] (0.8,0) -- (3,0);
        
        % 偏振后的E场
        \foreach \x in {1.2, 1.8, 2.4} {
            \draw[E field] (\x, 0) -- (\x, 0.5);
        }
        \node at (2.1, 0.9) {线偏振光};
    \end{tikzpicture}
    \caption{偏振片作用示意图}
\end{figure}

\begin{notice}{电磁波样式}
    \begin{itemize}
        \item \texttt{E field}, \texttt{B field} - 电/磁场向量
        \item \texttt{wave curve} - 波形曲线
        \item \texttt{polarizer}, \texttt{wave plate} - 光学元件
        \item 环境: \texttt{emwave}, \texttt{optics}
    \end{itemize}
\end{notice}

% ═══════════════════════════════════════════════════════════════════════════
\section{水印与批注系统}
% ═══════════════════════════════════════════════════════════════════════════

\subsection{水印功能}

\begin{notice}{水印命令}
    \begin{itemize}
        \item \verb|\mfWatermarkText{DRAFT}| - 默认灰色文字水印
        \item \verb|\mfWatermarkText[red!15]{绝密}| - 自定义颜色
        \item \verb|\mfWatermarkText[blue!10][6]{内部}| - 颜色 + 缩放
        \item \verb|\mfWatermarkImage{figure/logo.png}| - 图片水印
    \end{itemize}
    水印默认不显示,调用上述命令后才启用。
\end{notice}

\subsection{批注命令}

配合 \texttt{review} 选项使用效果更佳:

\todo{待完成:增加更多示例}

\fixme{需要修正:公式编号格式}

\notebox{备注:此功能在 v2.0 版本新增}

% ═══════════════════════════════════════════════════════════════════════════
\section{列表环境}
% ═══════════════════════════════════════════════════════════════════════════

\subsection{无序列表}

\begin{itemize}
    \item 第一级项目
    \begin{itemize}
        \item 第二级项目
        \begin{itemize}
            \item 第三级项目
        \end{itemize}
    \end{itemize}
    \item 另一个一级项目
\end{itemize}

\subsection{有序列表}

\begin{enumerate}
    \item 初始化模型参数 $\param$
    \item 前向传播计算预测值
    \item 计算损失函数 $\loss(\param)$
    \item 反向传播计算梯度 $\nabla_{\param} \loss$
    \item 更新参数:$\param \leftarrow \param - \eta \nabla_{\param} \loss$
\end{enumerate}

\subsection{方框编号列表}

\begin{enumsquared}
    \item 三种文档模式适应不同场景
    \item 十种 Section 样式自由切换
    \item 完整的数学定理环境
    \item 深度学习专用符号宏
    \item 美观的提示框与代码块
    \item 高级图文混排环境
\end{enumsquared}

% ═══════════════════════════════════════════════════════════════════════════
\section{数学公式}
% ═══════════════════════════════════════════════════════════════════════════

\subsection{行内与行间公式}

梯度下降更新规则:$\param_{t+1} = \param_t - \eta \nabla \loss(\param_t)$。

反向传播的链式法则:
\begin{equation}
    \pdv{\loss}{\weight^{(l)}} = \pdv{\loss}{\vect{a}^{(l)}} \cdot \pdv{\vect{a}^{(l)}}{\vect{z}^{(l)}} \cdot \pdv{\vect{z}^{(l)}}{\weight^{(l)}}
\end{equation}

\subsection{多行公式}

Adam 优化器的更新规则:
\begin{align}
    \vect{m}_t &= \beta_1 \vect{m}_{t-1} + (1 - \beta_1) \nabla \loss(\param_t) \\
    \vect{v}_t &= \beta_2 \vect{v}_{t-1} + (1 - \beta_2) (\nabla \loss(\param_t))^2 \\
    \hat{\vect{m}}_t &= \frac{\vect{m}_t}{1 - \beta_1^t}, \quad 
    \hat{\vect{v}}_t = \frac{\vect{v}_t}{1 - \beta_2^t} \\
    \param_{t+1} &= \param_t - \frac{\eta}{\sqrt{\hat{\vect{v}}_t} + \epsilon} \hat{\vect{m}}_t
\end{align}

% ═══════════════════════════════════════════════════════════════════════════
\section{总结}
% ═══════════════════════════════════════════════════════════════════════════

\begin{summarybox}[MindFlow 功能总览]
    \texttt{mindflow.cls} 提供了以下核心功能:
    \begin{enumsquared}
        \item \textbf{三种文档模式}: note / book / report
        \item \textbf{十种 Section 样式}: 基础 4 种 + 极客 6 种
        \item \textbf{定理环境}: 原版 + 美化版 tcolorbox
        \item \textbf{提示框}: notice / tip / warning / conclusion
        \item \textbf{图文混排}: textfigure / parallelfigures / figurerow / figuregrid
        \item \textbf{代码环境}: codeblock 语法高亮
        \item \textbf{数学宏库}: PDE + 泛函分析 + 深度学习
        \item \textbf{辅助功能}: 水印 / 批注 / 科学绘图
    \end{enumsquared}
\end{summarybox}

\end{document}
