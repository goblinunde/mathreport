% ╔═══════════════════════════════════════════════════════════════════════════╗
% ║                        MindFlow 功能演示文档                               ║
% ║                              Book 模式                                     ║
% ╠═══════════════════════════════════════════════════════════════════════════╣
% ║  文件: mathbook.tex                                                        ║
% ║  说明: 演示 mindflow.cls 的书籍模式,适合系统性技术书籍、学位论文撰写      ║
% ║  编译: xelatex mathbook.tex                                                ║
% ╚═══════════════════════════════════════════════════════════════════════════╝
\documentclass[linux, book]{mindflow}
%
% ┏━━━━━━━━━━━━━━━━━━━━━━━━━━━━━━━━━━━━━━━━━━━━━━━━━━━━━━━━━━━━━━━━━━━━━━━━━━━┓
% ┃                           📋 类选项完整参考                               ┃
% ┗━━━━━━━━━━━━━━━━━━━━━━━━━━━━━━━━━━━━━━━━━━━━━━━━━━━━━━━━━━━━━━━━━━━━━━━━━━━┛
%
% ┌─────────────────────────────────────────────────────────────────────────┐
% │ 【文档模式】三选一,决定文档结构                                         │
% ├─────────────────────────────────────────────────────────────────────────┤
% │  note   │ 基于 ctexart  │ 笔记模式,使用 \section 作为顶层              │
% │  book   │ 基于 ctexbook │ 书籍模式 ✓,支持 \chapter + \section          │
% │  report │ 基于 ctexrep  │ 报告模式,\section 呈现类似章节的大标题        │
% └─────────────────────────────────────────────────────────────────────────┘
%
% ┌─────────────────────────────────────────────────────────────────────────┐
% │ 【平台适配】三选一,影响字体加载                                         │
% ├─────────────────────────────────────────────────────────────────────────┤
% │  linux ✓ │ Fandol 字体族 (兼容性最佳)                                   │
% │  mac     │ macOS 系统字体 (Songti SC, Heiti SC)                        │
% │  win     │ Windows 系统字体 (SimSun, SimHei)                           │
% └─────────────────────────────────────────────────────────────────────────┘
%
% ┌─────────────────────────────────────────────────────────────────────────┐
% │ 【Section 样式】十选一,定制 \section 标题外观 (book 模式主要用于子节)   │
% ├───────────────────────┬─────────────────────────────────────────────────┤
% │   基础样式             │                                                 │
% ├───────────────────────┼─────────────────────────────────────────────────┤
% │  secstyle-classic     │  蓝色编号块 + 浅蓝背景条 (默认)                 │
% │  secstyle-modern      │  大号编号 + 渐变底边线装饰                      │
% │  secstyle-minimal     │  极简风格:左侧竖线 + 纯文本                    │
% │  secstyle-boxed       │  完整边框卡片 + 投影阴影                        │
% ├───────────────────────┼─────────────────────────────────────────────────┤
% │   极客样式             │                                                 │
% ├───────────────────────┼─────────────────────────────────────────────────┤
% │  secstyle-neon        │  🟢 霓虹发光:深色底 + 青绿光边                 │
% │  secstyle-terminal    │  🖥️ 终端风格:黑底绿字 + 命令提示符             │
% │  secstyle-gradient    │  🌈 渐变背景:蓝紫渐变 + 白色文字               │
% │  secstyle-elegant     │  ✨ 金色典雅:装饰线 + 居中排版                 │
% │  secstyle-blueprint   │  📐 蓝图网格:深蓝底 + 坐标标记                 │
% │  secstyle-ribbon      │  🎗️ 折叠丝带:红色丝带 + 阴影层次               │
% └───────────────────────┴─────────────────────────────────────────────────┘
%
% ┌─────────────────────────────────────────────────────────────────────────┐
% │ 【编号与审稿】                                                           │
% ├─────────────────────────────────────────────────────────────────────────┤
% │  chapnum   │ 公式/图表按章编号 (默认,如 式1.3, 图2.1)                  │
% │  nochapnum │ 全局连续编号 (如 式1, 式2...)                              │
% │  review    │ 启用行号显示                                               │
% └─────────────────────────────────────────────────────────────────────────┘
%
% ┏━━━━━━━━━━━━━━━━━━━━━━━━━━━━━━━━━━━━━━━━━━━━━━━━━━━━━━━━━━━━━━━━━━━━━━━━━━━┓
% ┃                            📌 特殊命令速查                                ┃
% ┗━━━━━━━━━━━━━━━━━━━━━━━━━━━━━━━━━━━━━━━━━━━━━━━━━━━━━━━━━━━━━━━━━━━━━━━━━━━┛
%
% 【目录】 \mftableofcontents  → 目录罗马数字,正文从1开始
% 【水印】 \mfWatermarkText[颜色][缩放]{文字} / \mfWatermarkImage{路径}
% 【批注】 \todo{...} / \fixme{...} / \notebox{...}
% 【术语】 \nomenclature{符号}{说明}  配合 \printnomenclature 使用
%
% ═══════════════════════════════════════════════════════════════════════════

\title{深度学习原理与实践}
\author{MindFlow Author}
\date{\today}

% 添加术语表条目
\nomenclature{$\mathcal{L}$}{损失函数 (Loss Function)}
\nomenclature{$\theta$}{模型参数 (Model Parameters)}
\nomenclature{$\eta$}{学习率 (Learning Rate)}
\nomenclature{CNN}{卷积神经网络 (Convolutional Neural Network)}
\nomenclature{RNN}{循环神经网络 (Recurrent Neural Network)}

\begin{document}

\maketitle
\mftableofcontents  % 目录使用罗马数字页码

% =========================================================
\chapter{神经网络基础}
% =========================================================

\begin{introbox}[本章导读]
    本章介绍神经网络的基本构成单元,包括感知机模型、激活函数选择、以及反向传播算法的数学原理。理解这些基础是深入学习深度学习的前提。
\end{introbox}

\section{感知机与激活函数}

神经网络的基本单元是感知机。给定输入 $\vect{x} \in \R^n$,输出由下式给出:
\[
    y = \sigma(\weight \vect{x} + \bias)
\]
其中 $\weight$ 是权重,$\bias$ 是偏置,$\sigma$ 是激活函数。

\begin{defnnew}{激活函数}
    激活函数 $\sigma: \R \to \R$ 为神经网络引入非线性。常见选择包括:
    \begin{itemize}
        \item Sigmoid: $\sigmoid(x) = \frac{1}{1+e^{-x}}$
        \item ReLU: $\relu(x) = \max(0, x)$
        \item Tanh: $\tanhfunc(x) = \frac{e^x - e^{-x}}{e^x + e^{-x}}$
    \end{itemize}
\end{defnnew}

\begin{parallelfigures}{常见激活函数可视化}
    \addfig[0.45]{figure/py2.png}{ReLU: $\max(0, x)$}
    \addfig[0.45]{figure/py3.png}{Sigmoid: $1/(1+e^{-x})$}
\end{parallelfigures}

\begin{remarknew}
    ReLU (Rectified Linear Unit) 因其计算简单且缓解了梯度消失问题,已成为现代深度网络的默认选择。但 ReLU 存在"死亡神经元"问题,Leaky ReLU 和 GELU 是常见替代方案。
\end{remarknew}

\section{反向传播算法}

反向传播 (Backpropagation) 是训练神经网络的核心算法。其本质是链式法则的高效实现。

\begin{theoremnew}{链式法则}
    设 $y = f(u)$ 且 $u = g(x)$,则:
    \[
        \pd{y}{x} = \pd{y}{u} \cdot \pd{u}{x}
    \]
    推广到向量形式,若 $\vect{y} = f(\vect{u}), \vect{u} = g(\vect{x})$,则 Jacobian 矩阵满足:
    \[
        J_{\vect{y}}(\vect{x}) = J_{\vect{y}}(\vect{u}) \cdot J_{\vect{u}}(\vect{x})
    \]
\end{theoremnew}

\begin{algorithm}[H]
\caption{计算图反向传播}
\begin{algorithmic}[1]
    \State 对计算图进行拓扑排序
    \For{每个节点 $v_i$ (按前向顺序)}
        \State 计算 $v_i$ 的输出值
    \EndFor
    \State 初始化输出梯度: $\bar{y} = 1$
    \For{每个节点 $v_i$ (按反向顺序)}
        \State 计算局部梯度 $\pd{v_j}{v_i}$
        \State 累积梯度: $\bar{v}_i = \sum_{j \in \text{Children}(v_i)} \bar{v}_j \pd{v_j}{v_i}$
    \EndFor
\end{algorithmic}
\end{algorithm}

\begin{summarybox}[本章小结]
    \begin{enumerate}
        \item 感知机是神经网络的基本单元,由线性变换和非线性激活组成
        \item 激活函数的选择影响网络的表达能力和训练稳定性
        \item 反向传播是链式法则的高效实现,计算复杂度与前向传播相同
    \end{enumerate}
\end{summarybox}

% =========================================================
\chapter{卷积神经网络 (CNN)}
% =========================================================

\begin{introbox}
    卷积神经网络是处理网格结构数据(如图像)的利器。本章介绍卷积运算的数学定义、CNN 的核心组件、以及经典架构的演进历程。
\end{introbox}

\section{卷积运算}

卷积层通过滑动窗口提取局部特征。二维离散卷积定义为:
\[
    (I * K)(i, j) = \sum_m \sum_n I(i+m, j+n) K(m, n)
\]
其中 $I$ 是输入图像,$K$ 是卷积核。

\begin{defbox}{CNN 关键参数}
    \begin{itemize}
        \item \textbf{Kernel Size ($k \times k$)}: 卷积核大小,决定感受野
        \item \textbf{Stride ($s$)}: 步长,控制输出特征图尺寸
        \item \textbf{Padding ($p$)}: 填充,保持边界信息完整性
        \item \textbf{Dilation ($d$)}: 膨胀率,扩大感受野而不增加参数
    \end{itemize}
    
    输出尺寸公式:$O = \lfloor \frac{I + 2p - d(k-1) - 1}{s} \rfloor + 1$
\end{defbox}

\section{多图排版演示}

\subsection{figurerow 环境}

\begin{figurerow}{CNN 卷积过程可视化}[3]
    \figitem{figure/py2.png}{输入图像}
    \figitem{figure/py3.png}{卷积核}
    \figitem{figure/py2.png}{特征图}
\end{figurerow}

\subsection{figuregrid 环境}

\begin{figuregrid}{ResNet 残差块结构}
    \gridrow{%
        \gridfig[0.45]{figure/py2.png}{标准残差块}
        \gridfig[0.45]{figure/py3.png}{瓶颈残差块}
    }
\end{figuregrid}

\section{现代架构演进}

从 LeNet 到 Transformer,视觉模型架构不断演进:

\begin{description}
    \item[LeNet-5 (1998)] 第一个成功的 CNN,用于手写数字识别
    \item[AlexNet (2012)] 引入 ReLU 和 Dropout,使用 GPU 训练,ImageNet 冠军
    \item[VGG (2014)] 使用小卷积核 ($3 \times 3$) 堆叠,证明深度的重要性
    \item[ResNet (2015)] 引入残差连接 $\vect{y} = \mathcal{F}(\vect{x}) + \vect{x}$
    \item[ViT (2020)] 将 Transformer 应用于图像,开启新范式
\end{description}

\begin{notebox}
    ResNet 的成功表明,让网络学习恒等映射 (Identity Mapping) 比学习零映射更容易。这一洞察启发了后续的 DenseNet、Highway Network 等架构。
\end{notebox}

\section{代码实现}

\begin{codeblock}[python]{PyTorch 残差块实现}
import torch.nn as nn

class ResidualBlock(nn.Module):
    def __init__(self, channels):
        super().__init__()
        self.conv1 = nn.Conv2d(channels, channels, 3, padding=1)
        self.bn1 = nn.BatchNorm2d(channels)
        self.conv2 = nn.Conv2d(channels, channels, 3, padding=1)
        self.bn2 = nn.BatchNorm2d(channels)
        self.relu = nn.ReLU(inplace=True)
    
    def forward(self, x):
        identity = x
        out = self.relu(self.bn1(self.conv1(x)))
        out = self.bn2(self.conv2(out))
        out += identity  # 残差连接
        return self.relu(out)
\end{codeblock}

\begin{summarybox}
    \begin{itemize}
        \item 卷积通过参数共享实现平移不变性
        \item 深度是 CNN 成功的关键因素
        \item 残差连接解决了深层网络的退化问题
    \end{itemize}
\end{summarybox}

% 术语表
\printnomenclature

\end{document}
