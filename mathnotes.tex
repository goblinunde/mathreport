% ╔═══════════════════════════════════════════════════════════════════════════╗
% ║                        MindFlow 功能演示文档                               ║
% ║                              Note 模式                                     ║
% ╠═══════════════════════════════════════════════════════════════════════════╣
% ║  文件: mathnotes.tex                                                       ║
% ║  说明: 演示 mindflow.cls 的核心功能,适合日常学术笔记与论文阅读记录        ║
% ║  编译: xelatex mathnotes.tex                                               ║
% ╚═══════════════════════════════════════════════════════════════════════════╝
\documentclass[linux, note, secstyle-terminal, chapnum]{mindflow}
%
% ┏━━━━━━━━━━━━━━━━━━━━━━━━━━━━━━━━━━━━━━━━━━━━━━━━━━━━━━━━━━━━━━━━━━━━━━━━━━━┓
% ┃                           📋 类选项完整参考                               ┃
% ┗━━━━━━━━━━━━━━━━━━━━━━━━━━━━━━━━━━━━━━━━━━━━━━━━━━━━━━━━━━━━━━━━━━━━━━━━━━━┛
%
% ┌─────────────────────────────────────────────────────────────────────────┐
% │ 【文档模式】三选一,决定文档结构                                         │
% ├─────────────────────────────────────────────────────────────────────────┤
% │  note   │ 基于 ctexart  │ 笔记模式 (默认),使用 \section 作为顶层      │
% │  book   │ 基于 ctexbook │ 书籍模式,支持 \chapter + \section 完整结构   │
% │  report │ 基于 ctexrep  │ 报告模式,\section 呈现类似章节的大标题样式   │
% └─────────────────────────────────────────────────────────────────────────┘
%
% ┌─────────────────────────────────────────────────────────────────────────┐
% │ 【平台适配】三选一,影响字体加载策略                                     │
% ├─────────────────────────────────────────────────────────────────────────┤
% │  linux  │ 使用 Fandol 字体族,兼容性最佳                                │
% │  mac    │ 使用 macOS 系统字体 (Songti SC, Heiti SC, Kaiti SC)          │
% │  win    │ 使用 Windows 系统字体 (SimSun, SimHei, KaiTi)                │
% └─────────────────────────────────────────────────────────────────────────┘
%
% ┌─────────────────────────────────────────────────────────────────────────┐
% │ 【Section 样式】十选一,定制章节标题外观                                 │
% ├───────────────────────┬─────────────────────────────────────────────────┤
% │      基础样式 (4)      │                  说明                           │
% ├───────────────────────┼─────────────────────────────────────────────────┤
% │  secstyle-classic     │  蓝色编号块 + 浅蓝背景条 (默认)                 │
% │  secstyle-modern      │  大号编号 + 渐变底边线装饰                      │
% │  secstyle-minimal     │  极简风格:左侧竖线 + 纯文本                    │
% │  secstyle-boxed       │  完整边框卡片 + 投影阴影                        │
% ├───────────────────────┼─────────────────────────────────────────────────┤
% │     极客样式 (6)       │                  说明                           │
% ├───────────────────────┼─────────────────────────────────────────────────┤
% │  secstyle-neon        │  🟢 霓虹发光:深色底 + 青绿光边 (赛博朋克)      │
% │  secstyle-terminal    │  🖥️ 终端风格:黑底绿字 + 命令行提示符           │
% │  secstyle-gradient    │  🌈 渐变背景:蓝紫渐变 + 白色文字               │
% │  secstyle-elegant     │  ✨ 金色典雅:装饰线 + 居中排版 (学术风)        │
% │  secstyle-blueprint   │  📐 蓝图网格:深蓝底 + 工程坐标标记            │
% │  secstyle-ribbon      │  🎗️ 折叠丝带:红色丝带装饰 + 阴影层次           │
% └───────────────────────┴─────────────────────────────────────────────────┘
%
% ┌─────────────────────────────────────────────────────────────────────────┐
% │ 【编号与审稿】                                                           │
% ├─────────────────────────────────────────────────────────────────────────┤
% │  chapnum   │ 公式/图表按章节编号 (如 式1-3, 图2-1)                      │
% │  nochapnum │ 全局连续编号 (默认,如 式1, 式2...)                        │
% │  review    │ 启用行号显示,方便审稿批注定位                             │
% └─────────────────────────────────────────────────────────────────────────┘
%
% ┏━━━━━━━━━━━━━━━━━━━━━━━━━━━━━━━━━━━━━━━━━━━━━━━━━━━━━━━━━━━━━━━━━━━━━━━━━━━┓
% ┃                            📌 特殊命令速查                                ┃
% ┗━━━━━━━━━━━━━━━━━━━━━━━━━━━━━━━━━━━━━━━━━━━━━━━━━━━━━━━━━━━━━━━━━━━━━━━━━━━┛
%
% 【目录】
%   \mftableofcontents     → 目录页使用罗马数字 (i, ii...),正文从1开始
%
% 【水印】(默认不显示,调用后启用)
%   \mfWatermarkText{DRAFT}              → 灰色文字水印
%   \mfWatermarkText[red!15]{绝密}       → 自定义颜色
%   \mfWatermarkText[blue!10][6]{内部}   → 颜色 + 缩放
%   \mfWatermarkImage{figure/logo.png}   → 图片水印
%
% 【批注】(配合 review 选项使用效果更佳)
%   \todo{内容}     → 橙色待办边框
%   \fixme{内容}    → 红色修正边框
%   \notebox{内容}  → 蓝色备注边框
%
% ═══════════════════════════════════════════════════════════════════════════

\title{MindFlow 笔记模式功能演示}
\author{MindFlow 用户}
\date{\today}

\begin{document}

\maketitle
\mftableofcontents  % 💡 使用此命令:目录用罗马数字,正文从1开始

% =========================================================
\section{数学宏库演示}
% =========================================================

\subsection{微分与积分算子}

MindFlow 提供丰富的数学宏,简化 PDE 和分析的排版:

\begin{itemize}
    \item 偏导数: $\pd{u}{t}$, $\pdd{u}{x}$ (二阶)
    \item 微分元: $\dx$, $\dy$, $\dt$, $\dmu$ (测度)
    \item 积分域: $\intO f \dx$, $\intRn f \dx$, $\intpO g \ds$ (边界)
\end{itemize}

热方程的经典形式:
\[
    \pd{u}{t} = \alpha \pdd{u}{x}, \quad x \in \Omega, \quad t > 0
\]

\subsection{函数空间与范数}

Sobolev 空间的嵌入定理:若 $u \in \Wkp{1}{p}(\Omega)$ 且 $p > n$,则 $u \in C(\bar{\Omega})$。

常用范数宏:
\begin{align}
    \normL{f} &:= \left( \intO |f|^p \dmu \right)^{1/p} \\
    \normL[2]{f} &= \sqrt{\intO |f|^2 \dmu}
\end{align}

\subsection{常用集合与符号}

\begin{itemize}
    \item 数集: $\R$ (实数), $\N$ (自然数), $\Z$ (整数), $\Q$ (有理数), $\C$ (复数)
    \item 收敛: $u_n \weakto u$ (弱收敛), $X \embeds Y$ (嵌入)
    \item 极限: $\limn a_n$, $\limx[0^+] f(x)$
\end{itemize}

% =========================================================
\section{定理环境演示}
% =========================================================

\subsection{经典 amsthm 环境}

\begin{theorem}[Lax-Milgram]
    设 $V$ 是 Hilbert 空间,$a(\cdot,\cdot)$ 是 $V$ 上的有界强制双线性形式,$F \in V'$。则存在唯一的 $u \in V$ 使得
    \[
        a(u, v) = F(v), \quad \forall v \in V
    \]
\end{theorem}

\begin{proof}
    利用 Riesz 表示定理构造算子 $A: V \to V$,证明其为双射。
\end{proof}

\subsection{TColorBox 美化环境}

\begin{defnnew}{Sobolev 空间 $W^{k,p}$}
    设 $\Omega \subset \R^n$ 是开集,$k \in \N$,$1 \le p \le \infty$。定义
    \[
        W^{k,p}(\Omega) := \Set{ u \in L^p(\Omega) \st D^\alpha u \in L^p(\Omega), \; |\alpha| \le k }
    \]
    其中 $D^\alpha u$ 是弱导数。
\end{defnnew}

\begin{theoremnew}{Poincaré 不等式}
    设 $\Omega$ 有界且有 Lipschitz 边界,$1 \le p < \infty$。则存在常数 $C > 0$ 使得
    \[
        \normL{u - \bar{u}} \le C \normL{\nabla u}, \quad \forall u \in W^{1,p}(\Omega)
    \]
    其中 $\bar{u} = \frac{1}{|\Omega|} \intO u \dx$ 是平均值。
\end{theoremnew}

\begin{proofnew}
    反证法:假设不存在这样的常数,构造序列并利用紧嵌入定理导出矛盾。
\end{proofnew}

\begin{lemmanew}{Grönwall 引理}
    若 $y(t) \le a(t) + \int_0^t b(s) y(s) \ds$,则 $y(t) \le a(t) + \int_0^t a(s) b(s) e^{\int_s^t b(r) \dd r} \ds$。
\end{lemmanew}

\begin{corollarynew}{能量估计}
    对于抛物方程,解的 $L^2$ 范数随时间指数衰减。
\end{corollarynew}

\begin{remarknew}
    以上定理是 PDE 弱解理论的基石,广泛应用于变分问题。
\end{remarknew}

% =========================================================
\section{提示框环境}
% =========================================================

\begin{notice}{关于 physics 包}
    MindFlow 加载了 physics 包,但由于与 siunitx 冲突,\verb|\qty| 命令被重定义。如需使用 siunitx 的 \verb|\qty|,请参考文档。
\end{notice}

\begin{tip}{高效学习建议}
    \begin{enumerate}
        \item 先理解物理直觉,再追求数学严谨
        \item 多做具体例子,建立计算感觉
        \item 使用 MindFlow 定理环境整理笔记
    \end{enumerate}
\end{tip}

\begin{warning}{常见错误}
    在 Sobolev 嵌入定理中,临界指数 $p^* = \frac{np}{n-p}$ 在 $p = n$ 时失效,此时需要 BMO 空间的工具。
\end{warning}

\begin{conclusion}{本节要点}
    \begin{itemize}
        \item $L^p$ 空间是 Banach 空间,$L^2$ 是 Hilbert 空间
        \item Sobolev 嵌入定理描述了导数正则性与连续性的关系
        \item Poincaré 不等式是变分法的核心工具
    \end{itemize}
\end{conclusion}

% =========================================================
\section{代码环境演示}
% =========================================================

以下是使用 PyTorch 求解 Poisson 方程的 PINN 代码:

\begin{codeblock}[python]{Physics-Informed Neural Network}
import torch
import torch.nn as nn

class PINN(nn.Module):
    def __init__(self, layers):
        super().__init__()
        self.net = nn.Sequential(*[
            nn.Sequential(nn.Linear(layers[i], layers[i+1]), nn.Tanh())
            for i in range(len(layers)-1)
        ])
    
    def forward(self, x):
        return self.net(x)

# 训练循环
model = PINN([2, 64, 64, 1])
optimizer = torch.optim.Adam(model.parameters(), lr=1e-3)
\end{codeblock}

% =========================================================
\section{图文混排演示}
% =========================================================

\begin{textfigure}[right]{figure/py2.png}{深度学习训练曲线}
    神经网络的训练过程可以通过损失函数的下降曲线来监控。理想情况下,训练损失和验证损失应该同时下降并趋于稳定。
    
    如果验证损失在某个点开始上升而训练损失继续下降,则说明模型出现了过拟合。此时可以考虑:
    \begin{itemize}
        \item 增加训练数据
        \item 使用正则化 (L2, Dropout)
        \item 提前停止训练
    \end{itemize}
\end{textfigure}

% =========================================================
\section{深度学习宏演示}
% =========================================================

神经网络的目标是最小化损失函数:
\[
    \loss(\param) = \frac{1}{N} \sum_{i=1}^{N} \ell(f_{\param}(\vect{x}_i), y_i)
\]
其中 $\param$ 是模型参数,$\data = \{(\vect{x}_i, y_i)\}$ 是训练数据。

常用损失函数:
\begin{itemize}
    \item 均方误差: $\MSE = \expect[(y - \hat{y})^2]$
    \item 交叉熵: $\CE = -\sum_c y_c \log \hat{y}_c$
    \item KL 散度: $\KLdiv(p \| q) = \sum_x p(x) \log \frac{p(x)}{q(x)}$
\end{itemize}

\end{document}
